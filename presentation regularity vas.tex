\documentclass[french]{beamer}
\usepackage[utf8]{inputenc}

\usepackage[T1]{fontenc}
\usepackage{babel}

\usepackage{xcolor}
%\usepackage{enumitem}
\usepackage{xspace}
\usepackage{amsmath, amssymb}
\usepackage{amsthm}
%\usepackage{graphicx}

%\usepackage{tikz} %to draw automata
%\usetikzlibrary{automata, positioning, arrows}

%\setbeamertemplate{footline}[frame number]
\usetheme{Warsaw}
\expandafter\def\expandafter\insertshorttitle\expandafter{%
  \insertshorttitle\hfill%
  \insertframenumber\,/\,\inserttotalframenumber}
\beamertemplatenavigationsymbolsempty
%\setbeamertemplate{footline}[frame number]

\newcommand{\set}[2]{\left\{#1\mathrel{\left|\vphantom{#1}\vphantom{#2}\right.}#2\right\}}
\newcommand{\os}[1]{\left\{\mathinner{#1}\right\}}
\newcommand{\defeq}{\ensuremath{\stackrel{\textit{def}}{=}}}
\newcommand{\eqfin}{\ensuremath{=_\text{fin}}}
\let\union\cup
\let\inter\cap
\newcommand{\abs}[1]{\ensuremath{\displaystyle\left\lvert #1 \right\rvert}}
\let\vide\varnothing
\newcommand{\Min}{\textit{Min}}
\newcommand{\Max}{\textit{Max}}

\newcommand{\N}{\ensuremath{\mathbb{N}}}
\newcommand{\Z}{\ensuremath{\mathbb{Z}}}
\newcommand{\Nomega}{\ensuremath{\mathbb{N}_\omega}}
\newcommand{\indicatrice}[1]{\ensuremath{\mathds{1}_{#1}}}
\newcommand{\card}[1]{|#1|}

\newcommand{\petri}{réseau de Petri\xspace}
\newcommand{\fire}[2]{\ensuremath{#1 (#2\rangle}}
\newcommand{\lang}{\ensuremath{\mathcal{L}}}
\newcommand{\reach}{\ensuremath{\textit{Reach}}}
\newcommand{\cover}{\ensuremath{\mathcal{C}}}
\newcommand{\clover}{\textit{Clover}\xspace}
\newcommand{\trans}[2]{\ensuremath{\stackrel{#1}{\longrightarrow}_{#2}}}
\newcommand{\transZ}[1]{\ensuremath{\stackrel{#1}{\rightsquigarrow}}}

\newcommand{\vect}[1]{\ensuremath{\mathbf{#1}}}
%\newcommand{\action}[1]{\ensuremath{\mathbf{#1}}}
\newcommand{\rel}{\ensuremath{\equiv}}
\newcommand{\relGY}{\ensuremath{\equiv^\text{GY}_S}}
\newcommand{\ssi}{\ensuremath{\text{ ssi }}}
\newcommand{\equivaut}{\ensuremath{\Leftrightarrow}}
\newcommand{\xinit}{\ensuremath{\vect{x}_\text{init}}}
\newcommand{\valeur}[1]{\ensuremath{\overline{#1}}}
\newcommand{\inirat}{\mathcal{R}}
\newcommand{\unite}[1]{\ensuremath{\vect{u}_{#1}}}

\newcommand{\Jfin}[1]{J^\text{fin}_{#1}}
\newcommand{\Jinf}[1]{J^\text{inf}_{#1}}


\newcommand{\alain}[1]{\textcolor{blue}{#1}}

\newtheorem{proposition}{Proposition}
\newtheorem{syntax}{Syntax}
\newtheorem{semantic}{Semantic}
\newtheorem{summary}{Contributions}
%\newtheorem{corollary}{Corollaire}
%\newtheorem{example}{Exemple}

\let\oldemph\emph
\renewcommand{\emph}[1]{\oldemph{\color{blue}#1}}

%%%%%%%%%%%%%%%%%%%%%%%%%%%%%%%%%%%%%%%%%%%%%%%%%%%

\title[Decidability of Regularity for VAS]{Décidabilité de la Rationalité pour les VAS}
\author[Lucas Bueri - avec Alain Finkel (LMF)]{\textbf{Lucas Bueri}}
\institute{\normalsize Stage de recherche de M2 - du 22 mars au 30 juillet 2021\\
\vspace{10pt}
avec Alain Finkel\\
\textit{LMF, Université Paris-Saclay}}
\date{\textit{MPRI} \\ Jeudi 2 septembre 2021}

\begin{document}
 \maketitle
 
%\begin{frame}
%\frametitle{L'environnement de recherche}
%\end{frame}

\begin{frame}{Contexte}
    Avantages des VAS rationnels :
    \begin{itemize}
        \item Construction d'un \emph{automate fini équivalent} (taille au plus Ackermann)
        \item Ensemble des configurations \emph{semi-linéaire}
        \item Plusieurs problèmes deviennent \emph{décidables} (inclusion des langages / des ensembles d'accessibilité)
        \item Diminue la \emph{complexité} de certains problèmes (model-checking)
    \end{itemize}
    
    \vspace{3mm}
    Problème de la rationalité du \emph{langage des traces} :
    \begin{itemize}
        \item Prouvé \emph{décidable} dès 1977 par Valk et Vidal-Naquet, puis Ginzburg et Yoeli.
        \item Prouvé \emph{EXPSPACE-complet} par Blockelet et Schmitz, puis Demri
    \end{itemize}
\end{frame}


%%%%%%%%%%%%%%%%%%%%%%%%%%%%%%%%%%%%%%%%%%%%%%%%%%

\begin{frame}{Système d'Addition de Vecteurs}
\begin{definition}[Syntaxe]
Un $d-$VAS $S=(A,\lambda)$ est composé
\begin{itemize}
    \item d'un alphabet fini $A$ d'\emph{actions},
    \item d'un étiquetage $\lambda:A\to\Z^d$ injectif.
    On notera $\valeur{a}=\lambda(a)$.
\end{itemize}
\end{definition}

\begin{definition}[Sémantique]
Les \emph{configurations} de $S$ sont les vecteurs de $\N^d$.

L'action $a\in A$ est \emph{franchissable} 
à partir de $\vect{x}\in \N^d$ lorsque $\vect{x} + \valeur{a} \geq \vect{0}$.
Son franchissement donne $\vect{x}\trans{a}{S} \vect{y} = \vect{x} + \valeur{a}$.
\end{definition}
\end{frame}

\begin{frame}{Exemple d'un VAS}
\begin{definition}[Langage]
\begin{itemize}
    \item 
    $\lang(S,\vect{x})\defeq \set{w\in A^\ast} {\exists \vect{y}\in\N^d, \vect{x}\trans{w}{S}\vect{y}}$ est le \emph{langage} des séquences d'actions franchissables à partir de $\vect{x}$,
    
    \item $\reach(S,\vect{x})\defeq \set{\vect{y}\in\N^d} {\exists w\in A^\ast, \vect{x}\trans{w}{S} \vect{y}}$ est l'ensemble des \emph{configurations accessibles} à partir de $\vect{x}$. 
\end{itemize}
\end{definition}

\vspace{3mm}
Exemple : 
Soit $S = (\os{a,b,c}, \lambda, \xinit=(0,0,0))$ 

%dont les actions sont étiquetés par 
avec $\valeur{a}=(1,0,0)$, $\valeur{b}=(0,1,-1)$ et $\valeur{c}=(-1,-1,1)$.

\vspace{3mm}
$\lang(S)= a^\ast$ est rationnel, et $\reach(S)= \set{\vect{x}_n= (n,0,0)} {n\in\N}$.
\end{frame}


\begin{frame}{Clover et le graphe de couverture}
\begin{definition}[Arbre de Karp-Miller]
Nœud racine $\xinit \in\N^d$.
Pour chaque nœud $s$ étiqueté par $\vect{x}\in\Nomega^d$, si $\vect{x} \trans{a}{S} \vect{y}$, on crée une arête $a$ :
\begin{itemize}
    \item Si $s$ a un ancêtre $\vect{y}$, vers une nouvelle feuille $\vect{y}$ ;
    \item Si $s$ a un ancêtre $\vect{z} > \vect{y}$, vers un nouveau nœud $\vect{y'}$
    
    (on pose $\vect{y'}(i)= \omega$ si $\vect{y}(i)>\vect{z}(i)$, $\vect{y'}(i)= \vect{y}(i)$ sinon) ;
    \item Sinon, vers un nouveau nœud $\vect{y}$.
\end{itemize}
\end{definition}

\vspace{3mm}
\emph{Clover} est alors défini par
\begin{itemize}
    \item l'ensemble des feuilles de l'arbre de couverture ;
    \item l'ensemble des idéaux maximaux de $\cover(S) \defeq \downarrow \reach(S,\xinit)$.
\end{itemize}
\end{frame}

\begin{frame}{Une relation d'équivalence sur $\N^d$}
\begin{definition}[Relation sur les configurations de $S$] 
$\forall \vect{x},\vect{y}\in\N^d, \vect{x} \rel_S \vect{y} \ssi \lang(S,\vect{x}) = \lang(S,\vect{y})$
\end{definition}

\begin{definition}[Relation de Ginzburg et Yoeli] 
$\vect{x}\relGY\vect{y} \ssi \forall w\in A^\ast, \big( \vect{x} +\valeur{w}\in\reach(S) \equivaut \vect{y} +\valeur{w}\in\reach(S) \big)$
\end{definition}

\begin{proposition}
Si $\vect{x} \relGY \vect{y}$ alors $\vect{x} \rel_S \vect{y}$.
Mais $\rel_S \not\subseteq \relGY$.
\end{proposition}

Contre-exemple : 
si $m>n>0$, on a 
\begin{itemize}
    \item $\lang(S,\vect{x}_m) =\lang(S,\vect{x}_n) =\lang(S)$,
    \item mais $\vect{x}_m \not\relGY \vect{x}_n$ (prendre $w= b^{n+1}c^{n+1}$).
\end{itemize}
\end{frame}


\begin{frame}{Caractérisation pour la rationalité}
\begin{theorem}
On a équivalence entre
\begin{enumerate}
    \item $\lang(S)$ est rationnel
    \item $\reach(S)/\rel_S$ est fini
    \item $\exists k\in\N, \forall \vect{x},\vect{y}\in\N^d, 
\big( \xinit\trans{*}{S} \vect{x} \trans{*}{S} \vect{y}\implies
\vect{y}\geq \vect{x} -\vect{k} \big)$
\end{enumerate}
\end{theorem}

\vspace{5mm}
Méthode de Valk et Vidal-Naquet : 
chercher un \emph{circuit négatif} du graphe de couverture

Méthode corrigée de Ginzburg et Yoeli :
se ramener au problème de \emph{bornitude} d'un VAS pour chaque élément de Clover. 
%en isolant les coordonnées infinies.
\end{frame}

\begin{frame}{Preuve de la caractérisation}

\begin{corollary}
$S$ est non-rationnel ssi $\exists i\in I, \forall k\in\N, \exists \; \xinit \trans{u_k}{S} \vect{x}_k \trans{v_k}{S} \vect{y}_k \land v_k(i) \leq -k$.
\end{corollary}

\vspace{5mm}
On divise $v_k$ pour avoir $q_0(i) > q_1(i) > \dots > q_\ell(i) \in\N^d$ tels que $\vect{x_k} = q_0 \trans{*}{S} q_1 \trans{*}{S} \cdots \trans{*}{S} q_\ell = \vect{y_k}$.

\vspace{3mm}
%Plus $k$ est grand, 
Si $\reach(S)/\rel_S$ est fini : \hspace{5mm}
$k$ grand fournit $q_\alpha \rel_S q_\beta$.

On prend $q_\alpha \trans{v}{S} q_\beta$ et
on répète $v$ pour épuiser la coordonnée $i$.
%donnant $\valeur{v}(\Jfin{x_k}) =\vect{0}$
\end{frame}

%************************************

\begin{frame}{Rationalité structurelle}
\begin{corollary}[Rappel]
$\lang(S,\xinit) \notin Rat$ ssi
$\exists i\in I, \forall k\in\N, \exists \; \xinit \trans{u_k}{S} \vect{x}_k \trans{v_k}{S} \vect{y}_k \land v_k(i) \leq -k$.
\end{corollary}

\begin{definition}
$S$ est \emph{structurellement rationnel} ssi $\lang(S,\vect{x})$ est rationnel pour tout $\vect{x} \in \N^d$.
\end{definition}

\begin{theorem}[inspiré de Vidal-Naquet]
$S=(A,\lambda)$ n'est pas structurellement rationnel ssi il existe $u,v\in A^*$ tels que 
\begin{enumerate}
    \item $\valeur{u} > \vect{0}$,
    \item Pour tout $i \in I$, $\valeur{u}(i)=0$ implique $\valeur{v}(i) \geq 0$ ,
    \item Il existe $i\in I$ tel que $\valeur{v}(i) < 0$.
\end{enumerate}
\end{theorem}
    
\end{frame}

\begin{frame}{Preuve avec le graphe de couverture}
    Si pour $\xinit$ initial, on a
    $$\forall k\in\N, \exists \; \xinit \trans{u_k}{S} \vect{x}_k \trans{v_k}{S} \vect{y}_k \land v_k(i) \leq -k$$
    
    Pour les brins $\vect{z} \trans{w}{G} \vect{z'}$ ajoutant des $\omega$ en lisant $u_k$, on a
%    \begin{itemize}
%        \item $\valeur{w}(\Jfin{z'})=\vect{0}$
%        \item $\valeur{w}(\Jfin{z}\inter \Jinf{z'})\geq \vect{1}$
%        \item $\valeur{w}(\Jinf{z})$ quelconque
%    \end{itemize}
$$\valeur{w}(\Jfin{z'})=\vect{0} \hspace{10mm}
\valeur{w}(\Jfin{z}\inter \Jinf{z'})\geq \vect{1} \hspace{10mm}
\valeur{w}(\Jinf{z}) \text{ quelconque}$$

On les combinent pour avoir $u$ vérifiant $\valeur{u}(\Jfin{\vect{x_k}}) =\vect{0}$ et $\valeur{u}(\Jinf{\vect{x_k}}) \geq\vect{1}$.

\vspace{8mm}
On divise $v_k$ pour avoir $q_0(i) > q_1(i) > \dots > q_\ell(i) \in\Nomega^d$ tels que $\vect{x_k} = q_0 \trans{*}{G} q_1 \trans{*}{G} \cdots \trans{*}{G} q_\ell = \vect{y_k}$.

\vspace{3mm}
$k$ grand fournit $q_\alpha = q_\beta$, on prend $q_\alpha \trans{v}{G} q_\beta$ donnant $\valeur{v}(\Jfin{x_k}) =\vect{0}$

\end{frame}

\begin{frame}{Preuve sur un exemple}
\begin{theorem}[rappel]
%$S=(A,\lambda)$ n'est pas structurellement rationnel ssi il existe $u,v\in A^*$ tels que 
\begin{enumerate}
    \item $\valeur{u} > \vect{0}$,
    \item $\forall i \in I$, $\valeur{u}(i)=0 \implies \valeur{v}(i) \geq 0$ ,
    \item $\exists i\in I, \valeur{v}(i) < 0$.
\end{enumerate}
\end{theorem}

\begin{example}[pas structurellement rationnel]
Pour tout $k\in\N$, $(0,0,1) \trans{a^k}{S} (k,0,1) \trans{(bc)^k}{S} (0,0,1)$.

\vspace{2mm}
Prendre $u=a$ et $v=bc$ dans le théorème.
Dans le graphe de couverture :
$(0,0,1) \trans{a}{S} (\omega,0,1) \trans{b}{S} (\omega,1,0) \trans{c}{S} (\omega,0,1)$
avec $\valeur{bc}(1)= -1$
\end{example}

%Si $\forall k, \xinit \trans{u_k}{S} \vect{x}_k \trans{v_k}{S} \vect{y}_k$ avec $\valeur{v_k}(i) \leq -k$
\end{frame}

\begin{frame}{Complexité de la rationalité structurelle}
\begin{theorem}
Le  problème de la rationalité structurelle est dans co-NP pour les VAS.
\end{theorem}

On retrouve la satisfaisabilité d'une \emph{formule de Presburger} :

\vspace{3mm}
Donnée : $n$ vecteurs $\lambda(A)=\{\vect{a_1}, \vect{a_1}, ..., \vect{a_n}\} \subseteq \Z^d$

Question : Existe-t-il $x_1,x_2,...,x_n, y_1,y_2,...,y_n \in\N$ tels que
\begin{enumerate}
    \item $\Sigma_{i=1}^n x_i \vect{a_i} > \vect{0}$
    \item Pour tout $j \in I$, $\Sigma_{i=1}^n x_i \vect{a_i}(j)=0$ implique $\Sigma_{i=1}^n y_i \vect{a_i}(j) \geq \vect{0}$,
    \item  $\lnot(\Sigma_{i=1}^n y_i \vect{a_i}, \geq \vect{0})$.
\end{enumerate}
\end{frame}


\begin{frame}{Configurations maximales rationnelles}
\begin{definition}
$\vect{x} \in \N^d$ est \emph{rationnelle pour $S$} ssi $\lang(S,\vect{x})$ est rationnel.

$$\inirat(S) \defeq \set{\vect{x} \in \N^d} {\lang(S,\vect{x}) \in Rat}$$
\end{definition}

\begin{theorem}
$\overline{\inirat(S)}$ est clos par le haut et 
$\Min(\overline{\inirat(S)})$ est fini
\end{theorem}

Si $\vect{h}\in\N^d$, on transporte le franchissement $\vect{x} \trans{u}{S} \vect{y}_k \trans{v}{S} \vect{z}_k$ en

$\vect{x} +\vect{h} \trans{u}{S} \vect{y}_k +\vect{h} \trans{v}{S} \vect{z}_k +\vect{h}$.
\end{frame}

\begin{frame}{Calcul de $\inirat(S)$}
\begin{theorem}[Valk et Jantzen - 1985]
Soit $K \subseteq \N^d$ clos par le haut.
Alors $\Min(K)$ est calculable ssi

pour tout $\vect{x} \in \Nomega^d$, le prédicat $p_K(\vect{x}) = (\downarrow \vect{x} \cap K \neq \vide)$ est décidable.
\end{theorem}

\vspace{3mm}
Problème à résoudre alors
\begin{itemize}
	\item[$\bullet$] Donnée : un VAS $S$ et une $\omega$-configuration $\vect{x}\in\Nomega^d$ ;
	\item[$\bullet$] Question : $S$ admet une configuration non-rationnelle dans $\downarrow \vect{x}$.
\end{itemize}
\end{frame}

\begin{frame}{Pistes pour calculer $\Min(\overline{\inirat(S)})$}
\begin{enumerate}
    \item Redéfinir le graphe de Karp-Miller et Clover lorsque la configuration initiale varie ;
    \item Mélanger la rationalité classique et structurelle en distinguant les coordonnées finies et infinies de $\vect{x}$ ;
    \item Modifier le VAS pour se ramener à un problème de rationalité de VAS classique (et gérer autrement les omegas) ;
    \item Complexifier la formule de Presburger (rationalité structurelle) pour gérer les actions ne pouvant être interverties.
\end{enumerate}
\end{frame}

\begin{frame}{Conclusion et références}

\begin{summary}
\begin{enumerate}
    \item Caractérisation de la rationalité des VAS et correction de la preuve de Ginzburg et Yoeli
    \item Preuve de décision pour la rationalité structurelle des VAS
    \item Résultats pour structurer les configurations rationnelles, et pistes pour trouver ces configurations
\end{enumerate}
\end{summary}

\vspace{5mm}
\footnotesize{
\begin{itemize}
    \item R. Valk et G. Vidal-Naquet.
    Petri Nets and Regular Languages.
    \textit{Journal of Computer and System Sciences}, 1981
    
    \item A. Ginzburg et M. Yoeli.
	Vector addition systems and regular languages.
	\textit{J. Comput. Syst. Sci.}, 1980
	
	\item R. Valk et M. Jantzen.
	The Residue of Vector Sets with Applications to Decidability Problems in Petri Nets.
    \textit{Acta Informatica}, 1985
\end{itemize}
}
    
\end{frame}

\end{document}
