\documentclass[a4paper,final]{article}

\usepackage[utf8]{inputenc}
\usepackage[T1]{fontenc}
\usepackage[french]{babel}

%\usepackage[backend=biber,isbn=false,doi=false,eprint=false]{biblatex}
%\usepackage[hyperindex=true]{hyperref}
%\usepackage[babel=true]{microtype}
%\addbibresource{biblio.bib}

\usepackage{xcolor}
\usepackage{enumitem}
\usepackage{xspace}
\usepackage{amsmath}
\usepackage{amssymb}
\usepackage{amsthm}
\usepackage{dsfont}
\usepackage{array}
\usepackage[french]{algorithm2e}
\usepackage{stmaryrd}
\usepackage[normalem]{ulem}

\usepackage{geometry}
\geometry{hmargin=3cm,vmargin=2cm}

\theoremstyle{definition}
\newtheorem{Theorem}{Théorème}
\newtheorem{Definition}[Theorem]{Définition}
\newtheorem{Proposition}[Theorem]{Proposition}
\newtheorem{Corollary}[Theorem]{Corollaire}
\newtheorem{Lemma}[Theorem]{Lemme}
\newtheorem{Property}[Theorem]{Propriété}
\newtheorem{Example}[Theorem]{Exemple}
\newtheorem*{Remark}{Remarque}
\newtheorem*{Statement}{Énoncé}
%\newtheorem{Proof}{Preuve}

\let\oldphi\phi
\let\phi\varphi
\let\oldepsilon\epsilon
\let\epsilon\varepsilon
\let\leq\leqslant
\let\geq\geqslant

%================================================================

\newcommand{\alain}[1]{\textcolor{blue}{#1}}
\newcommand{\lucas}[1]{\textcolor{olive}{#1}}
\newcommand{\rayer}[1]{\textcolor{red}{#1}}
% \sout pour rayer

\newcommand{\set}[2]{\left\{#1\mathrel{\left|\vphantom{#1}\vphantom{#2}\right.}#2\right\}}
\newcommand{\os}[1]{\left\{\mathinner{#1}\right\}}
\newcommand{\defeq}{\ensuremath{\stackrel{\textit{def}}{=}}}
\newcommand{\eqfin}{\ensuremath{=_\text{fin}}}
\let\union\cup
\let\inter\cap
\newcommand{\abs}[1]{\ensuremath{\displaystyle\left\lvert #1 \right\rvert}}
\let\vide\varnothing
\newcommand{\Min}{\textit{Min}}
\newcommand{\Max}{\textit{Max}}

\newcommand{\N}{\ensuremath{\mathbb{N}}}
\newcommand{\Z}{\ensuremath{\mathbb{Z}}}
\newcommand{\Nomega}{\ensuremath{\mathbb{N}_\omega}}
\newcommand{\indicatrice}[1]{\ensuremath{\mathds{1}_{#1}}}
\newcommand{\card}[1]{|#1|}

\newcommand{\petri}{réseau de Petri\xspace}
\newcommand{\fire}[2]{\ensuremath{#1 (#2\rangle}}
\newcommand{\lang}{\ensuremath{\mathcal{L}}}
\newcommand{\reach}{\ensuremath{\textit{Reach}}}
\newcommand{\cover}{\ensuremath{\mathcal{C}}}
\newcommand{\clover}{\textit{Clover}}
\newcommand{\trans}[2]{\ensuremath{\stackrel{#1}{\longrightarrow}_{#2}}}
\newcommand{\transZ}[1]{\ensuremath{\stackrel{#1}{\rightsquigarrow}}}

\newcommand{\vect}[1]{\ensuremath{\mathbf{#1}}}
\newcommand{\action}[1]{\ensuremath{\mathbf{#1}}}
\newcommand{\rel}{\ensuremath{\equiv}}
\newcommand{\relGY}{\ensuremath{\equiv^\text{GY}_S}}
\newcommand{\ssi}{\ensuremath{\text{ ssi }}}
\newcommand{\equivaut}{\ensuremath{\Leftrightarrow}}
\newcommand{\xinit}{\ensuremath{\vect{x}_\text{init}}}
\newcommand{\valeur}[1]{\ensuremath{\overline{#1}}}
\newcommand{\inirat}{\mathcal{R}}
\newcommand{\unite}[1]{\ensuremath{\vect{u}_{#1}}}

\newcommand{\Jfin}[1]{J^\text{fin}_{#1}}
\newcommand{\Jinf}[1]{J^\text{inf}_{#1}}

%================================================================

\title{Décidabilité de la rationalité pour les WSTS}
\author{Lucas \textsc{Bueri}}
\date{Stage M2 - 2021}

\begin{document}

\maketitle

%=============================================================

\section{Introduction}

{\bf Context} les VASS et les  réseaux de Petri sont difficiles à analyser car de nombreux problèmes sont indécidables (comme l'inclusion des ensembles d'accessibilité ou l'inclusion des langages de deux VASS) et les problèmes décidables ont souvent une complexité élevée. Par exemple, l'accessibilité d'un état de contrôle (et de nombreux problèmes de model checking qui s'y réduisent) est EXPSPACE-complet tandis que l'accessibilté d'une configuration est Ackermann-complete \cite{DBLP:journals/corr/abs-2104-12695,DBLP:journals/corr/abs-2105-08551}. 

Savoir si le langage d'un VASS est rationnel est important car dans ce cas, on peut effectivement construire un automate fini équivalent (la taille de celui-ci peut atteindre Ackermann \cite{vavn81}) et en conséquence, certains problèmes indécidables pour les VASS deviennent décidables pour les VASS rationnels (comme l'inclusion des langages de deux VASS étiquetés sur un même alphabet) et des problèmes décidables comme la couverture et l'accessibilité ont une complexité moindre qui est XXX \alain{en fait je n'en sais rien :)}. La rationalité du langage d'un VASS avec configurations finales, étiqueté même sans le mot vide, ainsi que l'universalité sont indécidables \cite{giyo80}. On considerera donc seulement les langages de traces (toutes les configurations sont finales) de VASS étiquetés injectivement et on dira qu'un VASS est rationnel quand son langage de traces est rationnel. 

On sait depuis 1977, par deux énoncés différents, de Yoeli et Ginzbourg  \cite{giyo80} et de Valk et Vidal-Naquet \cite{DBLP:conf/tcs/ValkV77,giyo80} que le problème de la rationalité du langage des traces d'un VASS étiqueté de façon injective est décidable \cite{giyo80,vavn81}. Yoeli et Ginzbourg réduisent la rationalité d'un VAS $S$ au fait qu'un nombre fini de VAS, construits à partir de $Clover(S)$, sont bornés (Théorème 2, \cite{giyo80}). Valk et Vidal-Naquet ont montré (Théorème 5,  \cite{vavn81}) qu'un réseau de Petri $S$ n'est pas rationnel s'il existe un circuit élémentaire, étiqueté par une séquence à effet strictement négatif sur au moins une coordonnée contenant un $\omega$, dans le graphe de Karp-Miller de $S$. Cet énoncé demande la construction du graphe de Karp-Miller mais nous verrons qu'on peut raisonner seulement sur Clover qui peut être beaucoup plus petit que le graphe de Karp-Miller.
On sait aussi, par deux preuves différentes de Blockelet et Schmitz \cite{DBLP:conf/mfcs/BlockeletS11}, et de Demri \cite{DBLP:journals/jcss/Demri13}, que la complexité de la rationalité est EXPSPACE-complet. %	\cite{DBLP:conf/mfcs/BlockeletS11,DBLP:journals/jcss/Demri13}.

Un autre problème de rationalité a été étudié dans  \cite{vavn81}, à savoir de décider si un VASS est rationnel pour \emph{toute} configuration initiale (problème de la  \emph{rationalité structurelle}).
\alain{il n'est pas évident que l'algorithme implicite demande de décider l'accessibilité car il s'agit de prouver l'existence d'une séquence de marquages accessibles mais pas de prouver l'accessibilité de deux marquages donnés.} 


{\bf Contribution}
\begin{itemize}
\item Nous montrons que la preuve du (Théorème 2, \cite{giyo80}) est techniquement fausse bien qu'elle contienne les bonnes idées; les auteurs ont considéré une relation d'équivalence $\equiv_{YG}$ qui ne satisfait pas la propriété : $S$ est rationnel ssi  le quotient $Reach(S)/\equiv_{YG}$ est fini. Nous introduisons une autre relation d'équivalence $\equiv_S$ telle que: $S$ est rationnel ssi  le quotient $Reach(S)/\equiv_S$ est fini.
    \item  Le problème de la rationalité structurelle est énoncé décidable (\cite{vavn81}, Théorème 6) mais la partie non triviale de la preuve est seulement évoquée informelllement en une phrase. Nous donnons une preuve complète que ce problème est bien décidable  \alain{expliquer le principe de la preuve}. Nous reformulons la caractérisation de non-rationalité jusqu'à obtenir un problème NP-complet.  (la complexité du problème de la rationalité structurelle semble n'avoir jamais été étudiée). 
%\item on corrige la preuve du résultat principal de \cite{giyo80} qui est formellement incorrecte. 
%En effet, YG énoncent que $S$ est rationnel ssi le quotient d'une relation d'équivalence $\equiv_1$ est fini mais nous montrons que ceci est faux. 
  
    \item On étudie l'ensemble $R(S)$ des configurations rationnelles d'un VAS $S$ (une configuration $\vect{x} \in \N^d$ est \emph{rationnelle pour $S$} si $\lang(S,\vect{x})$ est rationnel) et on montre que $R(S)$ est un ensemble clos par le bas. On montre, En utilisant un résultat de Valk \& Jantzen \cite{vaja85}, on calcule une base finie $Y \subseteq \N^d$ du complémentaire (clos par le haut) de $R(S)$ et donc on obtient aussi un ensemble fini $Z \subseteq \Nomega^d$ qui représente cet ensemble clos par le bas. Cet algorithme est au pire Ackermannian.\alain{est-il Ack-dur ? EXSPSACE dur ?}
Une fois l'ensemble $Z$ calculé, on peut répondre en temps linéaire en la taille de $Z$ si $S$ est rationnel à partir d'une configuration $x_0 \in \N^d$; il suffit de comparer $x_0$ avec les éléments dans $Z$. Si il existe un élément $z \in Z$ tel que $x_0 \leq z$ alors $L(S,x_0)$ est rationnel sinon $L(S,x_0)$ n'est pas rationnel. 

\end{itemize}
%%%%%%%%%
\iffalse
%%%%%%%%%
\section{Réseaux de Petri}

Un \petri $N = (P,T,B,F,M_0)$ est la donnée de
\begin{itemize}
    \item un ensemble fini $P$ de $d$ emplacements,
    \item un ensemble fini $T$ de transitions,
    \item une fonction de coût $B: P\times T\to\N$,
    \item une fonction de production $F: P\times T\to\N$,
    \item un marquage initial $M_0: P\to\N$.
\end{itemize}

Les configurations sont les marquages $M: P\to\N$, aussi considérés comme les valeurs possibles de $d$ compteurs (vecteur de $\N^d$).
On peut déclencher la transition $t$ à partir du marquage $M$ si et
seulement si $M(p)\geq B(p,t)$ pour tout $p\in P$ (noté $M\geq B(\cdot,t)$).

On obtient alors un nouveau marquage $M'$ défini par $M' \defeq M+D(\cdot,t)$ où $D\defeq F-B$. 
$B$ représente donc le coût de la transition (le nombre de jetons requis et consommés dans chaque emplacement), et $F$ représente sa production (les jetons créés lors du déclenchement).

On notera $\fire{M}{t}$ lorsque $t$ peut se déclencher sur $M$, et $\fire{M}{t}M'$ si déclencher $t$ sur $M$ donne $M'$.
On étendra naturellement cette notation (ainsi que $B(p,\cdot)$ et $F(p,\cdot)$) aux séquences de transitions, ou mots $w\in T^\ast$.

Deux ensembles nous intéresseront alors : le langage $\lang(N)\defeq \set{w\in T^\ast} {\fire{M_0}{w}}$ du \petri et les configurations accessibles $\reach(N)\defeq \set{M':P\to\N}{\exists w\in T^\ast, \fire{M_0}{w}M'}$.

%%%%%%%%%
\fi
%%%%%%%%%
%----------------------------------------------------------------

\section{VAS}

\subsection{La structure}

Un \emph{système d'addition de vecteurs de dimension} $d\in\N$ ($d-$VAS) $S=(A,\lambda)$ est la donnée 
d'un alphabet fini $A$ muni d'un étiquetage $\lambda:A\to\Z^d$.
À chaque \emph{action} $a\in A$ est ainsi associé un unique vecteur $\lambda(a)\in \Z^d$, de telle manière à ce que deux actions ne soient pas associées au même vecteur.
Pour des raisons de lisibilité, on notera $\valeur{a}=\lambda(a)$.
\vspace{3mm}
% On pourra préciser ce qu'il se passe si deux actions sont étiquetés par le même vecteur.

Les \emph{configurations} de $S$ sont alors les vecteurs de $\N^d$ (à coordonnées positives), 
et chaque action $a\in A$ agit sur $\N^d$ en additionnant à la configuration courante le vecteur $\valeur{a}$ associé.
On a alors une transition entre $\vect{x}$ et $\vect{y}$ étiquetée par l'action $a$ lorsque $\vect{x} + \valeur{a} = \vect{y}$.

De manière équivalente, on dira que l'action $a\in A$ est \emph{franchissable} 
à partir de la configuration $\vect{x}\in \N^d$ lorsque $\vect{x} + \valeur{a} \geq \vect{0}$,
et son déclenchement aboutit à la configuration $\vect{y} = \vect{x} + \valeur{a}$ à travers la transition $(\vect{x},a,\vect{y})\in \N^d\times A\times \N^d$.
On notera $\vect{x}\trans{a}{S} \vect{y}$ lorsqu'un tel déclenchement est possible (ou simplement $\vect{x}\trans{a}{} \vect{y}$ s'il n'y a pas ambiguïté sur $S$).

\begin{center}
Par la suite, on notera $I \defeq \os{1,\dots,d}$ l'ensemble des coordonnées pour les configurations.
\end{center}

%On munit $\N^d$ (et $\Z^d$) de l'addition terme à terme, et de l'ordre usuel $\vect{x}\leq\vect{y}$ ssi $\forall i\in I, \vect{x}(i) \leq \vect{y}(i)$.

Lorsqu'une séquence d'actions $w= a_1\cdots a_k\in A^\ast$ permet d'aller de $\vect{x}$ à $\vect{y}$ par la séquence de transition $\vect{x}= \vect{x_0}\trans{a_1}{} \vect{x_1}\trans{a_2}{} \dots\trans{a_k}{} \vect{x_k}=\vect{y}$
(où $\vect{x_0},\dots,\vect{x_k} \in\N^d$ et $\vect{x_{i-1}} +\valeur{a_i} =\vect{x_i}$ pour tout $1\leq i\leq k$),
on dit que $w$ est franchissable à partir de $\vect{x}$, et qu'on a une \emph{exécution} $\rho :\vect{x}\trans{w}{S} \vect{y}$.\\
$\vect{y}$ est alors dit accessible à partir de $\vect{x}$.

De plus, en notant $\valeur{w}\defeq \sum^k_{i=1} \valeur{a_i}$ le vecteur associé à $w$, on obtient $\vect{x} +\valeur{w} = \vect{y}$.
Attention, cette égalité peut-être vérifiée même si $w$ n'est pas franchissable.
\vspace{3mm}

Nous allons étudier deux ensembles naturellement associés à un VAS $S=(A,\lambda)$ :
\begin{enumerate}
    \item 
    $\lang(S,\vect{x})\defeq \set{w\in A^\ast} {\exists \vect{y}\in\N^d, \vect{x}\trans{w}{S}\vect{y}}$ qui est le \emph{langage} des séquences d'actions franchissables à partir de la configuration $\vect{x}$,
    
    \item $\reach(S,\vect{x})\defeq \set{\vect{y}\in\N^d} {\exists w\in A^\ast, \vect{x}\trans{w}{S} \vect{y}}$ qui est l'ensemble des configurations \emph{accessibles} à partir de $\vect{x}$. 
\end{enumerate}

\vspace{2mm}
On choisira souvent une configuration initiale $\xinit\in\N^d$, qu'on pourra ajouter à la définition du VAS.
On pourra alors regarder $\lang(S) \defeq \lang(S,\xinit)$ le langage du VAS $S = (A,\lambda,\xinit)$, et $\reach(S) \defeq \reach(S,\xinit)$ son ensemble d'accessibilité.

% A déplacer ?
\begin{Definition}
Un VAS $S = (A,\lambda,\xinit)$ est \emph{rationnel} lorsque $\lang(S)$ est rationnel sur $A^\ast$.
\end{Definition}

\begin{Definition}
Soit $S = (A,\lambda)$ un $d-$VAS,
$J\subseteq I$ un sous-ensemble d'indices et $\vect{x}\in\N^d$ une configuration.
On dit que
\begin{itemize}
    \item $J$ est \emph{borné} pour $\vect{x}$ sur $S$ lorsque toute configuration accessible depuis $\vect{x}$ a ses coordonnées dans $J$ bornées :
    $$\exists k\in\N, \forall \vect{y}\in\reach(S,\vect{x}), \forall j\in J, \vect{y}(j)\leq k$$
    
    \item $J$ est \emph{borné inférieurement} pour $\vect{x}$ sur $S$ lorsque toutes les coordonnées de $J$ ne diminue pas plus qu'une certaine borne (même en augmentant les ressources initiales) :
    $$\exists k\in\N, \forall n\in\N, \forall \vect{y}\in\reach(S,\vect{x} + n\cdot\unite{J}), \forall j\in J, \vect{y}(j)\geq \vect{x}(j)+n-k$$
    où $\unite{J}$ désigne le vecteur valant $1$ aux coordonnées dans $J$, et $0$ ailleurs.
\end{itemize}
\end{Definition}

Enfin, on dénotera par $Rat$ l'ensemble des langages rationnels (sur un alphabet fini).
%----------------------------------------------------------------

\subsection{Clover et le graphe de couverture}

On aimerait avoir un meilleur aperçu des configurations accessibles, et notamment décrire de manière finie les capacités pour le VAS d'atteindre des configurations non-bornées.
Rappelons d'abord la notion d'idéal :

\begin{Definition}
Soit $(X,\leq)$ un ensemble ordonné et $E\subseteq X$ un sous-ensemble de $E$.
\begin{itemize}
    \item $E$ est dit \emph{dirigé} lorsque pour tous $x,y\in E$ il existe un $z\in E$ vérifiant $x\leq z$ et $u\leq z$.
    \item $E$ est dit \emph{clos par le bas} lorsqu'il est égal à sa clôture par le bas $\downarrow E \defeq \set{x\in X}{\exists y\in E, x\leq y}$.
    \item De la même façon, $E$ est dit \emph{clos par le haut} lorsqu'il est égal à sa clôture par le haut $\uparrow E \defeq \set{x\in X}{\exists y\in E, x \geq y}$.
    \item Enfin, $E$ est un \emph{idéal} s'il est dirigé et clos par le bas.
\end{itemize}
\end{Definition}

$(\N^d,\leq)$ a la particularité d'être un bel ordre, ce qui permet d'obtenir des représentations finies de ces sous-ensembles.
Les idéaux de $\N^d$ peuvent donc se voir comme des éléments de $\Nomega^d$, obtenu en étendant $\N$ en $\Nomega\defeq \N\union\os{\omega}$ de façon naturelle. % À préciser ?
Ainsi, un élément $\mathfrak{m}\in \Nomega^d$ représente l'idéal $\set{\vect{y}\in\N^d}{\vect{y} \leq \mathfrak{m}}$.
On a alors le résultat suivant :

\begin{Proposition}
\lucas{référence ?}
Soit $E \subseteq \N^d$ clos par le bas.
Alors $E$ est une union finie d'idéaux de $\N^d$.
\end{Proposition}

Par conséquent, tout sous-ensemble de $\N^d$ clos par le bas peut être représenté par un ensemble fini d'éléments de $\Nomega^d$.
%	\vspace{5mm}
Pour les sous-ensembles $E$ de $\N^d$ clos par le haut, on va regarder ses éléments minimaux.
Notons $\Min(E) \defeq \set{x\in E}{\forall y < x, y \notin E}$.

\begin{Proposition}%référence ?
\label{clos_haut_wqo}
Soit $E\subseteq \N^d$ clos par le haut.
Alors $\Min(E)$ est fini et $\uparrow \Min(E) = E$.
\end{Proposition}

On peut maintenant introduire l'ensemble de couverture $\cover(S) \defeq \downarrow \reach(S,\xinit)$ d'un VAS $S = (A,\lambda,\xinit)$.
Il est clos par le bas dans $(\N^d,\leq)$, donc il se décompose en une union finie d'idéaux.

%\alain{dire que $\uparrow \emptyset = \emptyset$)}

\begin{Definition}
Soit $S=(A,\lambda,\xinit)$ un $d-$VAS.
On définit $\clover(S) \subseteq \Nomega^d$ comme l'ensemble des idéaux maximaux inclus dans $\cover(S)$.
C'est aussi l'unique ensemble, de taille minimale, d'idéaux dont l'union représente $\cover(S)$.
\end{Definition}

%-----------

%Cela va nous permettre de représenter le graphe des configurations accessibles de manière finie (bien qu'il puisse exister une infinité de configurations accessibles).
\lucas{À revoir.}

Le \emph{graphe de couverture} a pour sommets des configurations de $\Nomega^d$ et pour arêtes des transitions du VAS, étiquetés par une action de $A$.
Il est obtenu en partant d'un sommet initial $s_0:\xinit$ étiqueté par la configuration initiale $\xinit\in \N^d$, puis par récurrence sur la profondeur des noeuds en indiquant les voisins des noeuds accessibles :

Pour chaque noeud $s:\vect{x}$ associé à la configuration $\vect{x}\in \Nomega^d$, on fait partir de $s$ autant d'arêtes que d'actions $a\in A$ qui sont franchissables à partir de $\vect{x}$.
Le sommet d'arrivée de l'arête associée à une action $a$ est déterminé ainsi :
\begin{itemize}
    \item Si $\vect{x}\trans{a}{} \vect{y}$ (déclencher $a$ aboutit à la configuration $\vect{y} = \vect{x} + \valeur{a}$) 
    et qu'il existe un sommet déjà existant $r:\vect{y}$ associé à cette configuration, alors on crée une arête étiquetée par $a$ de $s:\vect{x}$ vers $r:\vect{y}$ ;
    \item Si $\vect{x}\trans{a}{} \vect{y}$ et qu'il existe un ancêtre $r:\vect{z}$ de $s$ (c'est-à-dire tel qu'il existe une chemin dans le graphe déjà créé de $r$ à $s$) avec $\vect{y}>\vect{z}$, 
    alors on crée un nouveau sommet $s':\vect{y'}$ et une arête de $s:\vect{y}$ vers $s':\vect{y'}$ étiquetée par $a$, 
    où $\vect{y'}\in \Nomega^d$ est la configuration de coordonnées $\vect{y'}(i) = \vect{y}(i)$ pour les $1\leq i\leq d$ tels que $\vect{y}(i)=\vect{z}(i)$, et $\vect{y'}(i) = \omega$ si $\vect{y}(i)>\vect{z}(i)$ ;
    \item Si la configuration $\vect{y}$ atteinte n'est pas dans les cas précédents, on crée simplement un nouveau sommet $s':\vect{y}$ et une arête de $s$ à $s'$ étiquetée par $a$.
\end{itemize}

%----------------------------------------------------------------

\section{Une nouvelle preuve de décidabilité de la rationalité}

La preuve de décidabilité se divise en deux étapes.
Tout d'abord, on va donner une caractérisation mathématique équivalente à la rationalité.
On montrera ainsi qu'un VAS est rationnel si et seulement s'il existe une borne $k\in\N$ telle que si on peut accéder à la configuration $\vect{x}$, puis à $\vect{y}$, alors $\vect{y}$ reste au dessus de $\vect{x}-\vect{k}$
%ssi il existe une borne sur la décroissance possible des coordonnées des configurations.
($\vect{k}$ désignera le vecteur $(k,k,...,k)\in\N^d$).

%%%%%%%%%%%
\iffalse
%%%%%%%%%%%%%%%%%%%%%
\subsection{La relation d'équivalence de Ginzburg et Yoeli n'est pas d'index fini}

Ginzburg et Yoeli introduisent dans \cite{giyo80} une relation d'équivalence $\relGY$ sur les configurations et énoncent que $\lang(S)$ est rationnel si et seulement si $\relGY$ admet un nombre fini de classes d'équivalence dans $\reach(S)$ (\cite{giyo80}, Théorème 1).

S'il est vrai que $\reach(S)/\relGY$ fini implique que $\lang(S)$ est rationnel, la réciproque est fausse et nous donnerons un contre-exemple d'un langage $\lang(S)$ rationnel tel que $\reach(S)/\relGY$ est infini. 
Nous proposerons de reprendre l'idée de Ginzburg et Yoeli,
mais en définissant une autre relation d'équivalence pour laquelle on obtiendra cette fois-ci l'équivalence entre la rationalité du langage et le quotient fini selon cette relation.

\begin{Definition}[\cite{giyo80} section 3]
Soit $S=(A,\lambda,\xinit)$ un VAS. La relation $\relGY$ est définie pour tout $\vect{x},\vect{y} \in\reach(S)$ par : 
$$\vect{x}\relGY\vect{y} \ssi \forall w\in A^\ast, \big( \vect{x} +\valeur{w}\in\reach(S) \equivaut \vect{y} +\valeur{w}\in\reach(S) \big)$$
\end{Definition}
\alain{ça donne quoi si on définit $\relGY$ sur tout $\N^d$ ? Une classe de plus seulement ? Une infinité ?}

\begin{Remark}
$\relGY$ est une relation d'équivalence sur l'ensemble $\reach(S)$ des configurations accessibles.
\end{Remark}

On aurait envie d'obtenir un résultat similaire à celui de Nérode, à savoir dire que $\lang(S)$ est rationnel si et seulement si $\relGY$ admet un nombre fini de classes d'équivalence.
Cela est malheureusement faux, puisque pour $\vect{x}\in\reach(S)$ et $w\in A^\ast$, l'écriture $\vect{x} +\valeur{w}\in\reach(S)$ ne permet pas de dire si la séquence $w$ est franchissable à partir de $\vect{x}$.
Il pourrait en effet exister une autre séquence $w'\in A^\ast$ franchissable à partir de $\vect{x}$ aboutissant à la configuration $\vect{x} +\valeur{w'} = \vect{x} +\valeur{w}$,
voire même un moyen d'accéder à la configuration $\vect{x} +\valeur{w} = \xinit +\valeur{u}$ depuis la configuration initiale par une autre séquence d'action $u\in A^\ast$ sans que $\xinit +\valeur{u}$ ne soit accessible depuis $\vect{x}$.

Plus précisément sur la preuve de Ginzburg et Yoeli, 
avoir $\reach(S)/\relGY$ fini implique bien $\lang(S)$ rationnel, ce qui est prouvé en construisant explicitement l'automate.
Par contre, la réciproque est fausse : 
L'erreur (avant-dernière ligne de la preuve du théorème 1 de \cite{giyo80}) était d'affirmer que savoir $\xinit +\valeur{uw}\in\reach(S)$ pour $u\in\lang(S)$ et $w\in A^\ast$ permettait d'en déduire que $uw\in\lang(S)$.

\vspace{5mm}

On donne ci-dessous un contre-exemple pour illustrer ce point.
Il est nécessaire de se placer au moins en dimension 3, car le résultat de Ginzburg et Yoeli reste vrai en dimension inférieure.

\lucas{Ajouter preuve que le résultat reste vrai en dimension inférieure à 2.}

\begin{Example}
Soit le $3-$VAS $S = (A=\os{a,b,c}, \lambda, \xinit=(0,0,0))$ dont les actions sont étiquetés par $\valeur{a}=(1,0,0)$, $\valeur{b}=(0,1,-1)$ et $\valeur{c}=(-1,-1,1)$.
Le langage reconnu $\lang(S)=a^\ast$ est rationnel, et les configurations accessibles sont les $\vect{x}_n=(n,0,0)$ pour $n\in\N$.

Cependant, pour deux entiers $m>n>0$, bien que $\lang(S,\vect{x}_m) =\lang(S,\vect{x}_n) =\lang(S)$, on a $\vect{x}_m \not\relGY \vect{x}_n$ :
Cela se constate en considérant la séquence d'actions $b^{n+1}c^{n+1}$ qui n'est jamais franchissable, mais qui vérifie $\vect{x}_m +\valeur{b^{n+1}c^{n+1}} = (m-n-1,0,0)\in \reach(S)$ alors que $\vect{x}_n +\valeur{b^{n+1}c^{n+1}} = (-1,0,0)\notin \reach(S)$.

La relation $\relGY$ admet alors une infinité de classes d'équivalences $(\os{\vect{x}_n})_{n\in\N}$ sur $S$.
\end{Example}
%%%%%%%%%%%%%%%%%%%%%
\fi
%%%%%%%%%%%%%%%%%%%%%

\subsection{Une nouvelle relation d'équivalence sur $\N^d$}
%	 qui est d'index fini (utilisée en 4.3 puis en 7)
%	Pour corriger ce problème, on va aussi regarder si les actions sont franchissables :

\begin{Definition}
Soit $S=(A,\lambda)$ un VAS. On introduit la relation $\rel_S$ sur les configurations en posant pour tout $\vect{x},\vect{y} \in\N^d$ :
$$\vect{x} \rel_S \vect{y} \ssi \lang(S,\vect{x}) = \lang(S,\vect{y})$$
\end{Definition}

% Regarder la relation d'ordre associée $x \sqsubseteq  y$ ssi $\lang(S,x) \subseteq \lang(S,y)$

Constatons déjà que cette nouvelle relation est incluse dans celle de Ginzburg et Yoeli :

\begin{Proposition}
Soit $S=(A,\lambda,\xinit)$ un VAS et $\vect{x},\vect{y} \in\lang(S)$.
Si $\vect{x} \relGY \vect{y}$ alors $\vect{x} \rel_S \vect{y}$.
\end{Proposition}

\begin{proof}
Supposons $\vect{x} \relGY \vect{y}$ et montrons $\lang(S,\vect{x}) \subseteq \lang(S,\vect{y})$ par récurrence sur la longueur des mots.
Soit $w\in\lang(S,\vect{x})$.

Si $w=\epsilon$ est le mot vide, $\vect{y} \in\lang(S)$ assure que $\epsilon\in\lang(S,\vect{y})$.

Sinon, on écrit $w=ua$ avec $u\in A^\ast$ et $a\in A$.
$u\in\lang(S,\vect{x})$ est plus court que $w$, donc par hypothèse de récurrence on a également $u\in\lang(S,\vect{y})$.
$u$ est donc franchissable depuis $\vect{y}$.
Mais $ua\in\lang(S,\vect{x})$, ce qui assure que $\vect{x} +\valeur{ua}\in\reach(S)$.

Comme $\vect{x} \relGY \vect{y}$, on obtient que $\vect{y} +\valeur{ua} \in\reach(S)$, aboutissant à $\vect{y} +\valeur{ua} \geq\vect{0}$.
L'action $a$ est donc franchissable depuis $\vect{y} +\valeur{u}$.
En résumé, on a les transitions valides $\vect{x} \trans{u}{S} \vect{x} +\valeur{u} \trans{a}{S} \vect{x} +\valeur{w}$, d'où $w\in\lang(S,y)$.

On conclut enfin que $\lang(S,\vect{x}) = \lang(S,\vect{y})$ par symétrie.
\end{proof}

On va établir le lien avec la relation de Nérode $\sim_L$ associée à un langage $L \subseteq A^*$. 
Pour tout $u,v\in A^\ast$, on définit: 
$$ u\sim_L v \ssi \forall w\in A^\ast, uw\in L \equivaut vw\in L $$

%$$\forall u,v\in A^\ast, \Big( u\sim_L v \ssi \forall w\in A^\ast, uw\in L \equivaut vw\in L \Big)$$

On sait que $\sim_L$ est une relation d'équivalence invariante par composition à droite et qu'un langage $L \subseteq A^*$ est rationnel si et seulement si $A^*/\sim_L$ est fini (\cite{rasc59}, Théorème 2).

%	admet un nombre fini de classes d'équivalences.

La congruence de Nérode concerne donc les mots plutôt que les configurations, mais est liée à l'équivalence $ \rel_S$ sur les VAS de la manière suivante :

\begin{Lemma}\label{lien avec Nérode}
    Soient $S=(A,\lambda,\xinit)$ un VAS et $u,v\in\lang(S)$.
    On a $u\sim_{\lang(S)}v$ si et seulement si $\xinit +\valeur{u} \rel_S \xinit +\valeur{v}$.
\end{Lemma}

\begin{proof}
%L'implication $u\sim_{\lang(S)}v \implies \xinit +\valeur{u} \rel_S \xinit +\valeur{v}$ est vrai pour tout $u,v \in A^*$.
Si $u\in\lang(S)$, alors pour tout mot $w\in A^\ast$, on a l'équivalence :
$$uw\in\lang(S) \equivaut w\in\lang(S,\xinit +\valeur{u})$$

On en déduit immédiatement le résultat en reprenant les définitions de chaque relation.
\end{proof}

%\alain{à relire et corriger}
%Remarque:  L'implication réciproque \alain{préciser quel sens:  $\xinit +\valeur{u} \rel_S \xinit +\valeur{v}$
%n'implique pas $u\sim_{\lang(S)}v$} si $u,v\notin\lang(S)$ \alain{est-ce suffisant qu'un des deux mots ne soit pas dans L(S) ? }, 
%auquel cas on a toujours $u\sim_{\lang(S)}v$.

\begin{Remark}
La relation de Nérode ne s'intéresse qu'aux mots du langage, et $\set{w\in A^\ast}{w\notin\lang(S)}$ forme une unique classe d'équivalence pour $\sim_{\lang(S)}$.
Ainsi, le lemme \ref{lien avec Nérode} devient faux dès lors que $u,v\notin\lang(S)$, puisque l'on a toujours $u\sim_{\lang(S)}v$ dans ce cas sans que $\xinit +\valeur{u} \rel_S \xinit +\valeur{v}$ ne soit nécessairement vrai.
\end{Remark}


\begin{Theorem}\label{lien relation-rationnel}
    Pour un VAS $S$, $\lang(S)$ est rationnel si et seulement si $\reach(S)/\rel_S$ est fini.
%	restreint aux configurations accessibles $\reach(S)$ admet un nombre fini de classes d'équivalence.
\end{Theorem}

\begin{proof}
On a les équivalences suivantes: \\
$\lang(S)$ est rationnel 
ssi $A^\ast/\sim_{\lang(S)}$ est fini (propriété de la relation de Nérode) \\
ssi $\lang(S)/\sim_{\lang(S)}$ est fini (car $\sim_{\lang(S)}$ admet un seule classe d'équivalence sur $A^\ast \setminus \lang(S)$) \\
ssi $\reach(S)/\rel_S$ est fini (par le Lemme \ref{lien avec Nérode}).
\end{proof}

Enfin, on montre que
%	donne une propriété de monotonie pour cette relation, qui appuie son intérêt pour l'étude du système de transition $S$. 
la relation d'équivalence $\rel_S$ 
%	sur les configurations d'un $d-$VAS $S$ 
est compatible/monotone avec les actions :

\begin{Proposition}\label{Monotonie relation}
Pour tout $\vect{x},\vect{y} \in\N^d$, $\vect{x}\rel_S \vect{y}$ implique $\forall a\in A, \vect{x} +\valeur{a} \rel_S \vect{y} +\valeur{a}$.
\end{Proposition}
%
%\begin{Remark}
%La relation $\relGY$ de Ginzburg et Yoeli vérifie également cette propriété.
%\end{Remark}

%--------------

\subsection{Borne sur la décroissance (décidable)}

Pour obtenir un nombre fini de classes d'équivalence pour $\rel_S$, on cherche une borne à partir de laquelle les configurations accessibles sont indiscernables.
Comme la seule règle restreignant les actions franchissables est un test de positivité, on va exiger que les configurations ne puissent pas trop décroître.\alain{cette explication est confuse}.

%	Nous suivons le raisonnement de Ginzburg et Yoeli mais sur notre relation d'équivalence $\equiv_2$ %	proposent une caractérisation au travers des deux lemmes suivants :
%\alain{ne plus parler de Yoeli: on en aura parlé dans l'intro et dans l'annexe. On ne les cite pas sur leur lemme 1 si on change la relation}

% Définir borné ?
\alain{écris ton lemme avec le maximum}
\begin{Statement}[\cite{giyo80} Lemme 1]\alain{à virer}
    Supposons que dans un VAS $S=(A,\lambda,\xinit)$, les $n$ premières coordonnées soient non-bornées.
    Supposons aussi qu'il existe $n$ entiers positifs $k_1,k_2,\dots,k_n$ tels que pour tout $\vect{x}\in\reach(S)$, tout $w\in A^\ast$ et tout $i=1,2,\dots,n$, 
    $(\vect{x} +\valeur{w})\in\reach(S)$ implique
    %si $w\in\lang(S,\vect{x})$ 
    $\vect{x}(i) - (\vect{x} +\valeur{w})(i) \leq k_i$.
    Alors $\reach(S)/\relGY$ est fini.
\end{Statement}

\xout{Ce résultat est correct, et nous l'adapterons facilement à la relation $\rel_S$ en modifiant la propriété requise en conséquence.
Notons qu'il n'est pas nécessaire de prendre des valeurs différentes pour les $k_i$, il est tout à fait possible de considérer leur maximum.}

\begin{Statement}[\cite{giyo80} Lemme 2]\alain{en annexe}
    Soit $S=(A,\lambda,\xinit)$ un VAS, et supposons qu'il existe une coordonnée non-bornée $j$ telle que 
    pour tout $k\geq 0$, il existe une configuration $\vect{x}\in\reach(S)$ et un mot $w\in A^\ast$ tels que 
    $(\vect{x} +\valeur{w}) \in\reach(S)$ et $\vect{x}(j) - (\vect{x} +\valeur{w})(j) > k$.
    Alors $\reach(S)/\relGY$ est infini.
\end{Statement}

\xout{Cette fois, la preuve donnée comporte une erreur de même nature que précédemment :
Il est affirmé que si $\vect{x} +\valeur{w} \in\reach(S)$, alors toutes les étapes intermédiaires sont accessibles, ce qui n'est pas forcément vrai.
Le résultat semble cependant vrai \lucas{(à vérifier)}, mais n'apporte pas la caractérisation souhaitée.}

\vspace{5mm}

\xout{Donnons maintenant une caractérisation similaire pour la relation $\rel_S$.
La preuve suit les idées de Ginzburg et Yoeli \cite{giyo80} en effectuant les modifications nécessaires.}\alain{on trouve ça dans Valk-Vidal-Naquet, le dire}

\begin{Theorem}[\cite{vavn81} théorème 3, \cite{giyo80}]\label{caractérisation}
    Soit $S=(A,\lambda,\xinit)$ un VAS.
    Alors $\lang(S)$ est rationnel si et seulement si
    \begin{equation}
        \exists k\in\N, \forall \vect{x},\vect{y}\in\N^d, 
\big( \xinit\trans{*}{S} \vect{x} \trans{*}{S} \vect{y}\implies
\vect{y}\geq \vect{x} -\vect{k} \big)
    \label{eq:caracterisation}
    \end{equation}
    %-----Une formulation en français-----:
    %il existe un $k\in\N$ tel que pour toutes configurations $\vect{x},\vect{y}\in\N^d$, si $\xinit\trans{*}{} \vect{x} \trans{*}{} \vect{y}$, alors $\vect{y}\geq \vect{x} -\vect{k}$
\end{Theorem}

\begin{proof}
Commençons par montrer que $S$ satisfait $ \eqref{eq:caracterisation} \Rightarrow \lang(S)$ rationnel.
Soit $S$ un VAS vérifiant la propriété \eqref{eq:caracterisation} pour un $k\in\N$.
%
Soit $\vect{x},\vect{y}\in\reach(S)$.
Supposons que $\vect{x}$ et $\vect{y}$ sont indiscernables pour les petites valeurs, c'est-à-dire que pour toute coordonnée $i\in I$, on a 
soit $\vect{x}(i)=\vect{y}(i)$, 
soit $\big( \vect{x}(i)\geq k \text{ et } \vect{y}(i)\geq k \big)$.
Alors $\vect{x}\rel_S \vect{y}$. 
En effet, on a $\vect{x} +\valeur{w} \geq\vect{0} \equivaut \vect{y} +\valeur{w} \geq\vect{0}$ pour tout $w\in A^\ast$, 
puisque les coordonnées qui diffèrent entre $\vect{x}$ et $\vect{y}$ ne peuvent devenir négatives.
Ainsi, $\rel_S$ admet au plus $(k+1)^d$ classes d'équivalences, donc $\lang(S)$ est rationnel (Théorème \ref{lien relation-rationnel}).

%	\vspace{4mm}\noindent
Prouvons maintenant, par contraposée, la réciproque $\lang(S)$ rationnel $\Rightarrow S$ vérifie $ \eqref{eq:caracterisation}$.
Si $S$ ne vérifie pas la propriété \eqref{eq:caracterisation}, alors
pour tout $k\in\N$, il existe une configuration accessible $\vect{x}$, un mot $w\in \lang(S,\vect{x})$ et une coordonnée $i\in I$ tels que $(\vect{x} +\valeur{w})(i)\leq \vect{x}(i)-k$.

On note $\vect{x_p} = \vect{x} +\valeur{a_1\cdots a_p} \in\reach(S)$ les différentes configurations obtenues en \xout{lisant} franchissant $w=a_1\cdots a_n$.
On a alors $\vect{x}=\vect{x}_0 \trans{a_1}{}\vect{x}_1 \trans{a_2}{}\dots \trans{a_n}{}\vect{x}_n= \vect{x} +\valeur{w}$.

Notons $\xi = \text{max}\set{\abs{\valeur{a}(j)}} {a\in A, j\in I}$ la valeur de la plus grande variation d'une coordonnée possible par une action.
Alors, au moins $k/\xi$ configurations $\vect{x}_p$ voient leur coordonnée $i$ décroître,
et l'on a une sous-séquence d'extractrice \alain{pas le bon mot}$\phi$ vérifiant $\vect{x}_{\phi(0)}(i)>\vect{x}_{\phi(1)}(i)>\dots>\vect{x}_{\phi(h)}(i)$ où $h\geq k/\xi$.


Ces configurations ne sont pas équivalentes pour $\rel_S$ :
En effet, si l'on avait $\vect{x}_{\phi(p)}\rel_S \vect{x}_{\phi(q)} = (\vect{x}_{\phi(p)} +\valeur{u})$ avec $0\leq p\leq q\leq h$ et en notant $u=a_{\phi(p)+1}\cdots a_{\phi(q)}$, 
%cela donne $\vect{x}_{\phi(p)} \rel_S (\vect{x}_{\phi(p)} +\valeur{u})$, d'où
alors on aurait $(\vect{x}_{\phi(p)} +\valeur{u^r}) \rel_S (\vect{x}_{\phi(p)} +\valeur{u^{r+1}})$ pour tout $r\in\N$ (en procédant par récurrence sur $r$ avec la Proposition \ref{Monotonie relation}).

Or $\valeur{u^r}(i) = r\times\valeur{u}(i) = r\times (\vect{x}_{\phi(q)}(i) -\vect{x}_{\phi(p)}(i)) < -r$, 
ce qui prouve que $(\vect{x}_{\phi(p)} +\valeur{u^r})(i)<0$ à partir d'un certain $r\in\N$, 
et donc $(\vect{x}_{\phi(p)} +\valeur{u^r}) \not\rel_S \vect{x}_{\phi(p)}$, d'où une contradiction.

On conclut qu'il existe au moins $k/\xi$ classes d'équivalences pour $\rel_S$ (et ce pour tout $k\in\N$), 
ainsi $\reach(S)/\rel_S$ est infini.
\end{proof}

%------------------------------------

\subsection{Décider la caractérisation (corrige Yoeli)}

\begin{Lemma}[\cite{giyo80} lemme 3]\label{mot décroissant}
    Soit $S=(A,\lambda,\xinit)$ un VAS et $k\in\N$.
    Supposons qu'il existe une configuration $\vect{x}\in \reach(S)$ et un mot $v\in\lang(S,\vect{x})$ franchissable tel que $\valeur{v}(i) < -k$ pour un certain $i \in I$.
    Alors on peut trouver une autre configuration $\vect{y}$ et un autre mot $w\in\lang(S,\vect{y})$ tel que $\valeur{w}(i) < -k$ et $\valeur{u}(i) \leq 0$ pour tout préfixe $u$ de $w$.
\end{Lemma}

\begin{proof}
Notons $z$ le plus long préfixe de $v$ tel que $\valeur{z}(i) \geq 0$.
On a alors $v = zw$, et le mot $w$ ainsi obtenu est franchissable à partir de $\vect{y} = \vect{x} + \valeur{z}$, 
et vérifie $\valeur{w}(i) = \valeur{v}(i) - \valeur{z}(i) \leq \valeur{v}(i) \leq k$.
De plus, pour tout préfixe $u$ de $w$, on a $\valeur{u}(i) = \valeur{zu}(i) - \valeur{z}(i) \leq \valeur{zu}(i) < 0$ puisque $zu$ est un préfixe de $v$ plus long que $z$.
\end{proof}

\begin{Definition}
Soient $S=(A,\lambda,\xinit)$ un VAS et $\mathfrak{m}\in\clover(S)$.
%\alain{pourquoi se mettre à dire d-VAS ? dire seulement VAS et que le d est implicite}
\xout{Pour tout \xout{idéal maximal} $\mathfrak{m}\in\clover(S)$} \xout{de l'ensemble de couverture,} notons $J_\mathfrak{m} \defeq \set{j}{\mathfrak{m}(j)\neq \omega}$ l'ensemble des coordonnées bornées pour les configurations de $\mathfrak{m}$.
On écrit $J_\mathfrak{m}=\os{j_1,\dots,j_r}$.

Pour tout $i\notin J_\mathfrak{m}$, on définit un $(r+1)-$VAS $S(\mathfrak{m},i) = \big( A(\mathfrak{m},i), \lambda(\mathfrak{m},i), \vect{x}_\mathfrak{m} \big)$ en posant
\begin{itemize}
    \item $A(\mathfrak{m},i) \defeq \set{a(\mathfrak{m},i)} {a\in A}$ 
    l'alphabet des actions
    \item $\lambda$ l'étiquetage qui à toute lettre $a(\mathfrak{m},i)$ associe le vecteur $\valeur{a(\mathfrak{m},i)} \defeq \big(\valeur{a}(j_1), \dots, \valeur{a}(j_r), -\valeur{a}(i) \big)$,
    \item $\vect{x}_\mathfrak{m} \defeq \big(\mathfrak{m}(j_1), \dots, \mathfrak{m}(j_r), 0\big) \in \N^{r+1}$ la configuration initiale.
\end{itemize}
\end{Definition}

\lucas{Revoir def, thm et preuve en gardant le même alphabet $A$, mais en changeant l'étiquetage.}

\noindent On regarde ainsi le comportement de chaque coordonnée non-bornée indépendamment \alain{expliquer} des autres.

On étend la correspondance entre les actions de $S$ et celles de $S(\mathfrak{m},i)$ aux mots : 
pour $w = a_1\cdots a_n \in A^\ast$, on pose $w(\mathfrak{m},i) \defeq a_1(\mathfrak{m},i) \;\cdots\; a_n(\mathfrak{m},i) \in A(\mathfrak{m},i)^\ast$.

Le théorème suivant est alors obtenu :
\alain{c'est exactement le théorème 2, car ici on parle de Clover, et YG parle des éléments maximaux de KM}

\begin{Theorem}[\cite{giyo80} théorème 2]
    Un VAS   $S=(A,\lambda,\xinit)$ est rationnel si et seulement si 
    tous les VAS $S(\mathfrak{m},i)$ sont bornés pour tout \xout{idéal maximal} $\mathfrak{m} \in \clover(S)$ et pour tout $i \in I \setminus J_\mathfrak{m}$.
%    \alain{ressemble à l'énoncé et la preuve de Valk Vidal-Naquet de 1977}
    %Soit $S=(A,\lambda,\xinit)$ un $d-$VAS.
    %Alors $S$ est rationnel si et seulement si 
    %tous les VAS $S(\mathfrak{m},i)$ sont bornés pour tout idéal maximal $\mathfrak{m} \in \clover(S)$ et tout $i \in I \setminus J_\mathfrak{m}$.
\end{Theorem}

\begin{proof}
Supposons que l'un des $S(\mathfrak{m},i)$ ne soit pas borné, et posons $r = \card{J_\mathfrak{m}}$.
La coordonnée $r+1$ (associée à $\mathfrak{m}(i)$) est la seule à pouvoir être non-bornée.
Pour tout $k\in\N$, il existe donc un mot $w(\mathfrak{m},i) \in \lang(S(\mathfrak{m},i))$ vérifiant $\valeur{w(\mathfrak{m},i)}(r+1) >k$.

Alors on peut trouver une configuration $\vect{x}\in\reach(S)$ de $\mathfrak{m}$ telle que $w\in\lang(S,\vect{x})$.
En effet, $w$ a le même effet sur les coordonnées $j_p \in J_\mathfrak{m}$ que $w(\mathfrak{m},i)$ sur $p\leq r$,
puisque pour tout $x\in\reach(S)$ et tout préfixe $u$ de $w$, on a
$(\vect{x} +\valeur{u})(j_p) = 
\mathfrak{m}(j_p) +\valeur{u}(j_p) = 
%(\vect{x}(\mathfrak{m},i) +\valeur{u(\mathfrak{m},i)})(p) = 
%\mathfrak{m}(j_p) +\valeur{u(\mathfrak{m},i)})(p) = 
\vect{x}_\mathfrak{m}(p) +\valeur{u(\mathfrak{m},i)})(p) \geq 0$.
Les autres coordonnées (non-bornées dans $\mathfrak{m}$) peuvent ensuite être choisies aussi grandes que nécessaire pour franchir $w$.

Comme $\valeur{w}(i) = -\valeur{w(\mathfrak{m},i)}(r+1) < -k$, on sait que $S$ ne vérifie pas \eqref{eq:caracterisation} pour ce $k$, 
ce qui assure que $\lang(S)$ n'est pas rationnel (par le théorème \ref{caractérisation}).

%----Réciproque
\vspace{3mm}
On suppose maintenant qu'il existe un entier $k\in\N$ majorant les coordonnées de toutes les configurations accessibles de tous les $S(\mathfrak{m},i)$.
Par l'absurde, supposons que la propriété \eqref{eq:caracterisation} n'est pas vérifiée pour ce $k$.
Par le lemme \ref{mot décroissant}, il existe une coordonnée $i\leq d$, une configuration $\vect{y}$ et un mot $w\in \lang(S,\vect{y})$ tel que  $\valeur{w}(i) < -k$ et $\valeur{u}(i) \leq 0$ pour tout préfixe $u$ de $w$.

Soit $\mathfrak{m} \in\clover(S)$ un élément maximal contenant $\vect{y}$.
Alors il existe une configuration $\vect{z}\in\mathfrak{m}$ vérifiant $\vect{y} \leq \vect{z}$.
Dans le VAS $S(\mathfrak{m},i)$, le mot $w(\mathfrak{m},i)$ appartient au langage $\lang(S(\mathfrak{m},i))$ 
puisque pour tout préfixe $u(\mathfrak{m},i)$ de $w(\mathfrak{m},i)$, on a 
\begin{itemize}
    \item $(\vect{x}_\mathfrak{m} +\valeur{u (\mathfrak{m},i)} )(p) = (\vect{z} +\valeur{u})(j_p) \geq (\vect{y} +\valeur{u})(j_p) \geq 0$ pour tout $p$ car $u$ est franchissable à partir de $\vect{y}$,
    \item $(\vect{x}_\mathfrak{m} +\valeur{u (\mathfrak{m},i)} )(r+1) = -\valeur{u}(i) \geq 0$ où $r = \card{J_\mathfrak{m}}$.
\end{itemize}
Néanmoins, la configuration accessible $(\vect{x}_\mathfrak{m} + \valeur{w(\mathfrak{m},i)}) \in \reach(S(\mathfrak{m},i))$ contredit l'hypothèse de borne puisque $\valeur{w(\mathfrak{m},i)}(r+1) = -\valeur{w}(i) > k$.

\end{proof}



La procédure de décision induite par le théorème XXX nécessite donc de connaître deux choses %	\alain{ce n'est pas argumenté donc ça n'explique rien, au contraire, à enlever}:
\begin{enumerate}
    \item $\clover(S)$ c'est-à-dire la liste des idéaux maximaux
    \item Une procédure pour décider si un VAS est borné.
\end{enumerate}
%--------------------------------------------------

\section{Décidabilité et complexité de la rationalité structurelle}

\lucas{J'ai mis cette partie avant de parler de configurations maximales, donc il y a des affirmations qui ne vont pas.}

Nous allons montrer qu'on peut décider dans NP si un VAS $S$ est rationnel pour tout $\vect{x} \in \N^d$. 

\begin{Definition}
On dit qu'un VAS $S$ est \emph{structurellement rationnel} si $\lang(S,\vect{x})$ est rationnel pour tout $\vect{x} \in \N^d$.
\end{Definition}

\begin{Remark}
On a donc que $S$ est structurellement rationnel si et seulement si $\inirat(S) = \N^d$ ou encore si
 $\Min(\overline{\inirat(S)})=\vide$. On observe que $S$ n'est jamais rationnel si $\Min(\overline{\inirat(S)})=\os{\vect{0}}$.  
%		$\lang(S,\vect{x})$ %(ou $(S,\vect{x})$ ?)  est rationnel pour tout  $\vect{x}\in\N^d$.
\end{Remark}

Nous allons étudier le problème de décision suivant appelé \emph{rationalité structurelle}:

Donnée : un $d-$VAS $S=(A,\lambda)$

Question : $S$ est-il structurellement rationnel ? \\

Pour $\vect{x},\vect{y}\in\Z^d$ (pas forcément positifs), on note $\vect{x} \transZ{u} \vect{y}$ pour $u\in A^*$ lorsque $\vect{y} = \vect{x} + \valeur{u}$.
Il s'agit ici d'une simple égalité vectorielle dans $\Z^d$, sans notion de franchissement (on pourra parler de $\Z$-VAS).
On écrira $\vect{x} \transZ{*} \vect{y}$ quand il existe $u\in A^*$ tel que $\vect{x} \transZ{u} \vect{y}$ .

Valk et Vidal-Naquet proposent dans \cite{vavn81} \alain{VV} une caractérisation pour la propriété de rationalité structurelle.

\begin{Theorem}[\cite{vavn81} théorème 6]
\label{rat_struct_valk}
Un $d-$VAS $S=(A,\lambda)$ n'est pas structurellement rationnel si et seulement s'il existe $\vect{x}_1, \vect{x}_2, \vect{x}_3, \vect{x}_4 \in \Z^d$ tels que 
\begin{enumerate}
    \item $\vect{x}_1 < \vect{x}_2$
    \item Pour tout $i \in I$, $\vect{x}_1(i) = \vect{x}_2(i)$ implique $\vect{x}_3(i) \leq \vect{x}_4(i)$,
    \item Il existe un $i\in I$ tel que $\vect{x}_3(i) > \vect{x}_4(i)$.
    
    \item $\vect{0} \transZ{*} \vect{x}_1 \transZ{*} \vect{x}_2 \transZ{*} \vect{x}_3 \transZ{*} \vect{x}_4$,
\end{enumerate}
\end{Theorem}


Énoncé ainsi, décider la rationalité avec le théorème \ref{rat_struct_valk} \xout{ demande d'utiliser un algorithme pour résoudre l'accessibilité ce qui est maintenant connu comme Ackermann-dur (ref récentes).}
Mais on peut remarquer que lister les configurations rencontrées dans le point 4 ci-dessus est inutile pour la rationalité structurelle, puisqu'on peut augmenter au besoin la configuration initiale  pour permettre les étapes intermédiaires.
On propose ainsi un autre théorème qui ne demande pas de décider l'accessibilité et - mieux - qui sera dans NP.
\alain{Reste à prouver la NP-dureté, c'est dommage si ce n'est pas fait.}

%---THM-STRUCT-RAT---

\begin{Lemma}\label{lem_access_generator}
    Soit $S=(A,\lambda,\xinit)$ un VAS initialisé et $\vect{x}\in\Nomega^d$ une $\omega$-configuration apparaissant dans le graphe de couverture $G$ de $S$.
    %Si $\vect{x}$ est $\omega$-accessible à partir de $\xinit$, 
    Alors il existe un mot $u\in A^*$ tel que
    \begin{enumerate}
        \item pour tout $j\in\Jfin{\vect{x}}$, on a $\valeur{u}(j) \geq 0$,
        \item pour tout $j\in\Jinf{\vect{x}}$, on a $\valeur{u}(j) \geq 1$.
    \end{enumerate}
\end{Lemma}

\begin{proof}
%On raisonne dans le graphe de couverture $G$ de $S$.
Comme $\vect{x}$ étiquette un sommet de $G$, il existe un mot $w \in A^*$ permettant d'accéder à $\vect{x}$ depuis $\xinit$ dans $G$.
Considérons la séquence des sommets du graphe parcourue en lisant le mot $w$, 
que l'on notera $\xinit = \vect{x_0} \trans{a_1}{G} \vect{x_1} \trans{a_2}{G} \cdots \trans{a_n}{G} \vect{x_n} = \vect{x}$ pour $w = a_1 a_2 \dots a_n$.

Les omegas apparaissant dans une de ces $\omega$-configurations ne peuvent disparaître au cours de l'exécution.
On a donc les inclusions $\vide = \Jinf{\vect{x_0}} \subseteq \Jinf{\vect{x_1}} \subseteq \cdots \subseteq \Jinf{\vect{x_n}}$.
Maintenant, pour chaque étape $1 \leq k \leq n$, 
%si $\Jinf{\vect{x_k}} \inter \Jfin{\vect{x_{k-1}}} \neq \vide$, %inutile
par construction du graphe, il existe un entier $0 \leq \ell \leq k$ vérifiant :
\begin{itemize}
    \item pour tout $j \in \Jfin{\vect{x_k}}$, on a $\vect{x_\ell}(j) = \vect{x_k}(j)$, % se résume en $\vect{x_\ell} \eqfin \vect{x_k}$
    \item pour tout $j \in \Jinf{\vect{x_k}} \inter \Jfin{\vect{x_{k-1}}}$, on a $\vect{x_\ell}(j) < \vect{x_k}(j)$.
\end{itemize}
On a donc l'existence d'un mot $w_k = a_{\ell+1} \dots a_k$ vérifiant
$\valeur{w_k}(j)=0$ pour tout $j \in \Jfin{\vect{x_k}}$, et
$\valeur{w_k}(j)>0$ pour tout $j \in \Jinf{\vect{x_k}} \inter \Jfin{\vect{x_{k-1}}}$.

\vspace{2mm}
Par récurrence sur $k \leq n$, on construit un mot $u_k$ vérifiant 
$\valeur{u_k}(j) = 0$ pour tout $j \in \Jfin{\vect{x_k}}$, et
$\valeur{u_k}(j) > 0$ pour tout $j \in \Jinf{\vect{x_k}}$.
Prendre le mot vide pour $u_0$ convient puisque $\Jinf{\xinit} = \vide$.

Soit $1 \leq k \leq n$.
Si un $u_{k-1}$ convenable existe, notons $\delta = - \min \set{\valeur{w_k}(j)}{j \in \Jinf{\vect{x_{k-1}}}}$ la plus grande valeur négative dans une coordonnée de $\valeur{w_k}$.
Posons ensuite $u_k = {(u_{k-1})}^{\delta + 1} \, w_k$, obtenu en concaténant $\delta + 1$ fois le mot $u_{k-1}$ au mot $w_k$.
On a alors
\begin{itemize}
    \item pour tout $j \in \Jfin{\vect{x_k}}$, $\valeur{u_k}(j) = \valeur{u_{k-1}}(j) \times (\delta + 1) + \valeur{w_k}(j) = 0$ puisque $\Jfin{\vect{x_{k-1}}} \subseteq \Jfin{\vect{x_k}}$,
    \item pour tout $j \in \Jinf{\vect{x_k}}$, $\valeur{u_k}(j) = \valeur{u_{k-1}}(j) \times (\delta + 1) + \valeur{w_k}(j) \geq \delta + 1 - \delta > 0$.
\end{itemize}
ce qui conclut l'hérédité de la récurrence.

\vspace{2mm}
Ainsi, le mot $u = u_n$ satisfait le lemme.
\end{proof}

%%%%%%%%%%%%%%
On notera que le mot $u$ ainsi construit n'est pas forcément franchissable à partir de $\xinit$.
Néanmoins, la rationnalité structurelle permet d'éviter cette contrainte.

\begin{Theorem}\label{conf_struct_rat}
\label{rat_struct_nouveau}
Un $d-$VAS $S=(A,\lambda)$ n'est pas structurellement rationnel si et seulement s'il existe $u,v\in A^*$ tels que 
\begin{enumerate}
    \item $\valeur{u} > \vect{0}$,
    \item Pour tout $i \in I$, $\valeur{u}(i)=0$ implique $\valeur{v}(i) \geq 0$ ,
    \item Il existe $i\in I$ tel que $\valeur{v}(i) < 0$.
\end{enumerate}
\end{Theorem}

\begin{Remark}
Les caractérisations des théorèmes \ref{rat_struct_valk} et \ref{rat_struct_nouveau} sont bien équivalentes :
%\alain{(mal dit)}

\noindent À partir des configurations $\vect{x}_1, \vect{x}_2, \vect{x}_3, \vect{x}_4$, on choisit simplement pour $u$ le mot permettant d'aller de $\vect{x}_1$ à $\vect{x}_2$, et pour $v$ le mot allant de $\vect{x}_3$ à $\vect{x}_4$.

\noindent Dans l'autre sens, avec les mots $u,v$, on regarde simplement l'exécution $\vect{0} = \vect{x}_1 \transZ{u} \vect{x}_2 = \vect{x}_3 \transZ{v} \vect{x}_4$.
\end{Remark}

Prouvons maintenant le théorème \ref{rat_struct_nouveau}.
Commençons par le sens direct, et construisons deux mots $u$ et $v$ vérifiant les points 1, 2 et 3 lorsque le VAS $S$ n'est pas structurellement borné.

\begin{proof}
% Point de départ
Supposons que $S$ ne soit pas structurellement rationnel.
Il existe donc une configuration initiale $\xinit \in\N^d$ telle que $\lang(S,\xinit)$ n'est pas rationnel.
D'après le théorème \ref{caractérisation}, on sait que pour tout $k\in\N$,  il existe une séquence d'actions franchissable 
$\xinit \trans{u_k}{S} \vect{x}_k \trans{v_k}{S} \vect{y}_k$
et une coordonnée $i_k \in I$ tels que $\valeur{v_k}(i_k) \leq -k$.

% Définitions
Un fois la séquence d'action fixée, on notera $J\subseteq I$ l'ensemble des coordonnées $j$ gardant $\big( \valeur{u_k}(j) \big)_{k\in\N}$ borné.
%\alain{confus: coordonnées entières,  dans tous les $u_k$ donc il faut expliquer que cette phrase a un sens car le coordonnées finies pourraient, à priori bouger selon les k considérés}
Formellement, on définit 
$$J = \set{j\in I}{
    \text{sup} \set{\valeur{u_k}(j)}{k\in\N} \text{ est fini}}
    %\exists h_j\in\N, \forall k\in\N, \valeur{u_k}(j) \leq h_j}
\text{, et pour } j\in J,\hspace{2mm} h_j = \max \set{\valeur{u_k}(j)}{k\in\N}$$
%\alain{on peut définir J = j tel que sup{(uk(j)/k} est fini }

% Restrictions
Commençons par ajouter des propriétés supplémentaires à la suite $\big((u_k,v_k)\big)_{k\in\N}$.
Pour se faire, remarquons que si on remplace cette suite par une sous-suite $\big( (u_{\phi(k)}, v_{\phi(k)}) \big) _{k\in\N}$ extraite avec $\phi:\N\to\N$ strictement croissante, 
alors les mêmes propriétés sont vérifiées :
Pour tout $k\in\N$,
\begin{itemize}
    \item le mot $u_{\phi(k)} v_{\phi(k)}$ est franchissable à partir de $\xinit$, et on a $\xinit \trans{u_{\phi(k)}}{S} \vect{x}_{\phi(k)} \trans{v_{\phi(k)}}{S} \vect{y}_{\phi(k)}$,
    \item il existe une coordonnée $i_{\phi(k)} \in I$ telle que $\valeur{v_{\phi(k)}}(i_{\phi(k)}) \leq -\phi(k) \leq -k$.
\end{itemize}
Par conséquent, on peut se restreindre à une sous-suite, et ainsi sélectionner uniquement les séquences d'actions qui nous intéresse.

% Sélections
\begin{itemize}
    \item[$\bullet$] Comme $I$ est fini, les indices $i_k$ ne peuvent prendre qu'un nombre fini de valeurs possibles, donc (par le principe des tiroirs) il existe un $\mathrm{i}\in I$ apparaissant une infinité de fois.
    On choisit une extractrice $\phi$ telle que $i_{\phi(k)} = \mathrm{i}$ pour tout $k\in\N$.
    %On peut ainsi supposer que $i_k = \mathrm{i}$ pour tout $k$.
    
    \item[$\bullet$] Comme $(\N^d,\leq)$ forme un bel ordre, la suite $(\valeur{u_k})_{k\in\N}$ admet une sous-suite croissante pour $\leq$, que l'on sélectionne.
    %Là encore, à un renomage des indices près, on peut se ramener au cas où $(\valeur{u_k})_{k\in\N}$ est croissante.
\end{itemize}
On notera $J\subseteq I$ l'ensemble des coordonnées $j$ gardant $\big( \valeur{u_k}(j) \big)_{k\in\N}$ borné.
Formellement, on définit $J = \set{j\in I} {\text{sup} \set{\valeur{u_k}(j)} {k\in\N} \text{ est fini}}$,
et pour $j\in J$, $h_j = \max \set{\valeur{u_k}(j)}{k\in\N}$.

 \begin{itemize}   
    \item[$\bullet$] Pour tout $j\in J$, comme $\big( \valeur{u_k}(j) \big)_{k\in\N}$ est croissante, l'ensemble $\set{k\in\N} {\valeur{u_k}(j) < h_j}$ est fini.
    Il existe donc un $k_0\in\N$ tel que pour tout $k\geq k_0$ et tout $j\in J$, on ait $\valeur{u_k}(j) = h_j$.
    On se restreint alors à la sous-suite obtenue en sélectionnant les $k\geq k_0$.
    
    \item[$\bullet$] Pour tout $j\in I-J$, la suite $\big( \valeur{u_k}(j) \big)_{k\in\N}$ est croissante et tend vers $+\infty$.
    %Les \xout{autres} coordonnées $\valeur{u_k}(j)$ pour $j\in I-J$ sont aussi croissantes \alain{une suite peut être croissante mais pas des coordonnées}et tendent vers l'infini \alain{on pourrait utiliser la notation, après definition, $lim_{k \rightarrow \infty} u_k$  et même $lim  \ u_k$}.
    On peut donc se restreindre aux $k\in\N$ tels que pour tout $j\notin J$, on ait $\valeur{u_k}(j) \geq k$.
\end{itemize}

\vspace{2mm}
% Utilisation du graphe de couverture
On va raisonner à l'aide du graphe de couverture de $S$ à partir de la configuration initiale $\xinit$.
Fixons un $k\in\N$ strictement plus grand que toutes les valeurs entières apparaissant dans les $\omega$-configurations du graphe de couverture, et regardons le comportement du mot $u_k$ sur ce graphe.

$u_k$ est bien franchissable à partir de $\xinit$, et son franchissement dans le graphe de couverture atteint une $\omega$-configuration $\vect{x}$.
Choisissons un élément maximal $\vect{x'}\in\clover$ tel que $\vect{x} \leq \vect{x'}$.
\lucas{On n'a pas unicité !}
Pour tout $j\in I-J$, on a $\valeur{u_k}(j) \geq k$, d'où $\vect{x'}(j) \geq \vect{x}(j) \geq k$, ce qui assure que $\vect{x'}(j) = \omega$ (grâce au le choix de $k$).
Ainsi, on a l'inclusion $I-J \subseteq \Jinf{\vect{x'}}$ (ce qui se réécrit en $\Jfin{\vect{x'}} \subseteq J$).

Maintenant, en utilisant le lemme \ref{lem_access_generator} sur $\vect{x'}$, on obtient l'existence d'un mot $u\in A^*$ tel que 
\begin{itemize}
    \item pour tout $j\in\Jfin{\vect{x'}}$, on a $\valeur{u}(j) \geq 0$,
    \item pour tout $j\in\Jinf{\vect{x'}}$, on a $\valeur{u}(j) > 0$.
\end{itemize}
En particulier, $\valeur{u} \geq \vect{0}$ (ce qui montre le point 1 du théorème), 
et pour tout $j \notin J$, $\valeur{u}(j) > 0$.
\vspace{4mm}

% Points 2 et 3 avec les mots v_k
Intéressons-nous maintenant aux mots $v_k$.
Posons $\xi = \max \set{\abs{\valeur{a}(i)}} {a\in A,i\in I}$, et notons $\gamma$ le nombre de sommets du graphe de couverture $G$ de $(S,\xinit)$.
En choisissant $k > \gamma \times \xi$, on peut factoriser $v_k$ en $\gamma + 1$ mots $w_i$ 
avec le franchissement de $v_k = w_1 w_2 \dots w_{\gamma + 1}$ à partir de $\vect{x_k}$ suivant :
$$\vect{x_k} = \vect{z_0} \trans{w_1}{S} \vect{z_1} \trans{w_2}{S} \cdots \trans{w_{\gamma + 1}}{S} \vect{z_{\gamma + 1}} = \vect{y_k}$$
de telle manière à ce que $\vect{z_0}(\mathrm{i}) > \vect{z_1}(\mathrm{i}) > \cdots > \vect{z_{\gamma + 1}}(\mathrm{i})$.
On rappelle que $\mathrm{i} \in I$ est la coordonnée telle que $\valeur{v_k}(\mathrm{i}) \leq k$.

% On repasse sur le graphe de couverture pour trouver un circuit
Le mot $v_k$ est franchissable à partir de $x_k$, donc à partir de la $\omega$-configuration $\vect{x}$, mais également à partir de $\vect{x'}$.
Ce franchissement se factorise ainsi :
$$\vect{x'} = \vect{q_0} \trans{w_1}{G} \vect{q_1} \trans{w_2}{G} \cdots \trans{w_{\gamma + 1}}{G} \vect{q_{\gamma + 1}}$$
où les $\vect{q_\ell}$ sont des $\omega$-configurations.
Soulignons que comme $\vect{x'}$ est maximal, on a $\Jinf{\vect{q_0}} = \cdots = \Jinf{\vect{q_{\gamma+1}}}$.
Par le lemme des tiroirs, deux de ces $\omega$-configurations $\vect{q_\ell}$ et $\vect{q_{\ell'}}$ sont identiques, et on obtient un circuit partant de la $\omega$-configuration $\vect{q_\ell}$ et étiqueté par un facteur $v\in A^*$ de $v_k$.

\vspace{2mm}
% On repasse aux configurations finies
En revenant au franchissement de $v_k$ à partir de $\vect{x_k}$, on a donc un facteur $v$ tel que $\vect{z_\ell} \trans{v}{S} \vect{z_{\ell'}}$ avec
%$\vect{z_\ell} \eqfin \vect{z_\ell'}$ et $\vect{z_\ell}(\mathrm{i}) > \vect{z_\ell'}(\mathrm{i})$.

\begin{itemize}
    \item $\vect{z_\ell} \eqfin \vect{z_\ell'}$ qui permet de montrer le point 2.
    En effet, pour $i\in I$, si $\valeur{u}(i) = 0$, alors $i\in\Jfin{\vect{x'}} = \Jfin{\vect{z_\ell}}$ (car $\vect{x'}$ est maximal), ce qui assure que $\valeur{v}(i) = 0$.
    
    \item $\vect{z_\ell}(\mathrm{i}) > \vect{z_\ell}(\mathrm{i})$ qui prouve que $\valeur{v}(\mathrm{i}) < 0$, montrant ainsi le point 3.
\end{itemize}
\end{proof}

%-+-+-+-+-+-+-+-+-+-+-+-+-

Prouvons maintenant le sens réciproque du théorème \ref{rat_struct_nouveau}.

\begin{proof}
Réciproquement, supposons qu'il existe deux mots $u,v\in A^*$ vérifiant les points 1, 2 et 3.
En choisissant des coordonnées suffisamment grandes, on peut trouver une configuration $\xinit \in \N^d$ à partir de laquelle le mot $u$ est franchissable.
\lucas{Est-ce plus clair ainsi ?}
Montrons que $\lang(S,\xinit)$ n'est pas rationnel.

Soit $k\in\N$.
Notons $\rho = \text{max}\set{-\valeur{v}(i)}{i\in I}$, 
et montrons alors que $u^{\rho\times k} v^k \in \lang(S,\xinit)$.

\begin{itemize}
    \item[$\bullet$] D'après le point 1, on a $\valeur{u} > \vect{0}$, donc $\xinit + \valeur{u} > \xinit$.
    Comme $u$ est franchissable à partir de $\xinit$, le mot $u^{\rho\times k}$ l'est également.
    Le franchissement $\xinit \trans{u^{\rho\times k}}{S} \vect{x}_k$ permet d'atteindre une configuration $\vect{x}_k \in\N^d$.
    
    \item[$\bullet$] D'après le point 2, si pour un $i\in I$, on a $\valeur{v^k}(i) < 0$, alors $\valeur{u}(i) > 0$, d'où $\vect{x}_k(i) \geq \rho\times k$.
    Par définition de $\rho$, on en déduit que $v^k$ est franchissable à partir de $\vect{x}_k$, et le franchissement donne $\vect{x}_k \trans{v^k}{S} \vect{y}_k$ avec $\vect{y}_k \in\N^d$.
    
    \item[$\bullet$] Enfin, d'après le point 3, il existe un $\mathrm{i}\in I$ tel que $\valeur{v}(\mathrm{i}) < 0$, d'où $\valeur{v^k}(\mathrm{i}) \leq -k$.
\end{itemize}

Par conséquent, la séquence $\xinit \trans{u^{\rho\times k}}{S} \vect{x}_k \trans{v^k}{S} \vect{y}_k$ vérifie la propriété \eqref{eq:caracterisation} pour ce $k$.
On notera que $\xinit$ est bien indépendant de $k$.
Le langage $\lang(S,\xinit)$ n'est donc pas rationnel, et $S$ n'est pas structurellement rationnel.
\end{proof}

\begin{Theorem}\label{NP conf_struct_rat}
Le  problème de la rationalité structurelle est dans NP pour les VAS.
\end{Theorem}

\begin{proof}
Le théorème \ref{rat_struct_nouveau} montre que le problème de la rationalité structurelle revient à décider le problème de décision suivant.

Donnée : un ensemble de vecteurs $\lambda(A) \subseteq \Z^d$ (les étiquettes des actions de $S = (A,\lambda)$).

Question : Existe-t-il deux vecteurs $\vect{u}$ et $\vect{v}$ obtenus comme combinaison linéaire à coefficients entiers positifs de $\lambda(A)$ vérifiant les 3 conditions suivantes:
\alain{finalement je ne comprends pas ces 3 conditions: je ne vois pas bien le lien entre u et v et j'ai envie de dire que si u existe c'est facile de trouver un v: supposons qu'on a trouvé un u tel que (1). on pose  v=u  Si ça marche on a juste besoin de supposer qu'il existe un u...satisfaisant (1)...}
\begin{enumerate}
    \item $\vect{u} > \vect{0}$, \alain{revient à tester d fois si $Ax\geq b$ avec A la matrice du VAS et b=(1,0,...0), b=(0,1,...0), b=(0,0...1)}
    \item Pour tout $i \in I$, $\vect{u}(i)=0$ implique \alain{ce n'est pas vraiment une implication logique, il suffit de prendre v=u presque partout} $\vect{v}(i) \geq 0$,
    \item  $\lnot(\vect{v} \geq 0)$. \alain{sauf sur un indice i (qui existe) où u(i)>0 et alors on pose v(i)=-u(i), vérifier que c'est possible en jouant sur les coefficients: prendre $\beta_j=-\alpha_j$}
\end{enumerate}
c'est donc dans \alain{co-NP-complet et pas nécessairement dans NP-complet}.
\end{proof}



\section{Rationnalité sur des $\omega$-configurations (Brouillon)}
\alain{dans le papier Valk 1977..., la rationalité est donnée comme plus tard dans Yoeli. 1980..}
%\textbf{Méthode de Valk et Vidal-Naquet :}

%On prouve d'abord la réciproque en construisant un automate fini reconnaissant $\lang(S)$.
%Pour cela, on restreint simplement les états aux configurations de $\os{0,\dots,c}^d$ pour une certaine constante $c\in\N$ obtenue à partir de $k$.
%L'autre sens est montré par l'absurde, en s'appuyant sur les circuits dans le graphe de couverture de $S$.

On étend les configurations de $\N^d$ à des $\omega$-configurations dans $\Nomega^d$.
On retrouve alors le lien avec les idéaux exposé dans la section 2.2.

Pour une $\omega$-configuration $\mathfrak{m} \in \Nomega^d$, on notera $\Jfin{\mathfrak{m}} = \set{i\in I}{\mathfrak{m}(i) \text{ est fini}}$ l'ensemble des coordonnées bornées et  $\Jinf{\mathfrak{m}} = \set{i\in I}{\mathfrak{m}(i)=\omega}$ l'ensemble des coordonnées non-bornées de $\mathfrak{m}$.

Une coordonnée valant $\omega$ dans une configuration signifie que l'on a autant de ressources que nécessaires.
ces coordonnées n'entrent donc pas en compte dans le franchissement.

\begin{Definition}
Soit $S = (A,\lambda)$ un VAS, $\vect{x} \in \Nomega^d$ une $\omega$-configuration et $u \in A^*$ une séquence d'actions.
On dit que $u$ est \emph{franchissable} à partir de $\vect{x}$ lorsque pour tout préfixe $v$ de $u$ et pour toute coordonnée $i \in \Jfin{\vect{x}}$, on a $\vect{x}(i) + \valeur{v}(i) \geq 0$.

La $\omega$-configuration obtenue en appliquant $u$ à $\vect{x}$ est alors $\vect{y} = \vect{x} + \valeur{u}$.
Ainsi, $\Jinf{\vect{x}} = \Jinf{\vect{y}}$.
\end{Definition}

Cette notion de franchissement ne modifie pas les coordonnées valant $\omega$, et agit sans en tenir compte.
Dans le cas où $\vect{x}$ a toutes ses coordonnées finies, on retrouve le franchissement classique dans $\N^d$.

Maintenant, pour permettre de générer des $\omega$ depuis une coordonnée finie, on va exiger de pouvoir augmenter cette coordonnée autant que nécessaire.
\alain{faire le lien avec la definition d'omega transition et d'acceleration dans papier avec Igor et Serge}

\begin{Definition}
Soit $S = (A,\lambda)$ un VAS et $\vect{x},\vect{y} \in \Nomega^d$ deux $\omega$-configurations.
On dit que $\vect{y}$ est \emph{$\omega$-accessible} à partir de $\vect{x}$ lorsque pour tout $k\in\N$, il existe un mot $u_k \in A^*$ franchissable à partir de $\vect{x}$ tel que 
\begin{enumerate}
    \item pour tout $i\in\Jfin{\vect{y}}$, on a $\vect{x}(i) + \valeur{u_k}(i) = \vect{y}(i)$
    \item pour tout $i\in \Jinf{\vect{y}} \inter \Jfin{\vect{x}}$, on a $\vect{x}(i) + \valeur{u_k}(i) \geq k$
\end{enumerate}
\end{Definition}

Ce processus correspond aux transitions du graphe de couverture.
On le retrouve sous le nom d'\emph{accélération} dans \cite{FHK-fossacs2020}, et de \emph{coordonnées non-bornées avec contexte \vect{x}} dans \cite{vavn81}.
On constate que cette notion correspond à l'accessibilité classique lorsque $\vect{x}$ et $\vect{y}$ sont des configurations (finies) de $\N^d$.


\begin{Theorem}[\cite{vavn81} théorème 4]
Soit $S$ un VAS.
Alors $\lang(S)$ n'est pas rationnel si et seulement si
il existe une $\omega$-configuration $\vect{y} \in \Nomega^d$ tel que
%non-borné sur les coordonnées de $J = \set{i\in I}{\mathfrak{m}(i)=\omega}$ tel que
\begin{enumerate}
    \item $\vect{y}$ est $\omega$-accessible maximal dans $S$ (c'est-à-dire $\omega$-accessible depuis $\xinit$ et il n'existe pas de configuration $\omega$-accessible plus grande dans $\Nomega^d$),
    \item $\Jinf{\vect{y}}$ n'est pas borné inférieurement pour la configuration $\vect{y'}\in\N^d$, 
    obtenue en posant $\vect{y'}(i) = 0$ si $i\in \Jinf{\vect{y}}$ et $\vect{y'}(i) = \mathfrak{i}$ si $i\in \Jfin{\vect{y}}$.
\end{enumerate}
\end{Theorem}

\begin{Theorem}[\cite{vavn81} lemme 3]
Soit $S$ un VAS.
Alors $\lang(S)$ n'est pas rationnel si et seulement si
il existe un circuit étiqueté par $v\in A^*$ dans le graphe de couverture partant d'une $\omega$-configuration $\omega$-accessible maximale $\vect{y}$, 
tel que $\valeur{v}(i) < 0$ pour une coordonnée $i\in \Jinf{\vect{y}}$.
\end{Theorem}

%---------------------

\begin{Definition}
Soient $S = (A,\lambda)$ un VAS, $\vect{x}\in\Nomega^d$ \xout{une $\omega$-configuration} et $v\in A^*$ \xout{un mot}.
On dit que $v$ est \emph{cyclable} à partir de $\vect{x}$ lorsque $v$ est franchissable à partir de $\vect{x}$, et qu'il permet d'accéder à la même $\omega$-configuration $\vect{x}$ (i.e. $\vect{x} \trans{v}{S} \vect{x}$).
\end{Definition}

\begin{Theorem}
Soit $S = (A,\lambda)$ un VAS et $\vect{x}\in\Nomega^d$ une $\omega$-configuration.
Alors $\lang(S,\vect{x})$ n'est pas rationnel si et seulement si
il existe une $\omega$-configuration $\vect{y}\in\Nomega^d$ et un mot $v\in A^*$ tels que
\begin{enumerate}
    \item $\vect{y}$ est $\omega$-accessible depuis $\vect{x}$ dans $S$,
    \item $v$ est cyclable à partir de $\vect{y}$,
    \item $\valeur{v} \not\geq \vect{0}$ (i.e. il existe un $i\in I$ tel que $\valeur{v}(i)<0$).
\end{enumerate}
\end{Theorem}


Si un mot est cyclable à partir de x, il l'est à partir de y>x.   une $\omega$-configuration plus grande.
On peut donc exiger que $\vect{y}$ soit maximal (donc dans Clover).

$\vect{x}$ $\omega$-configuration générable \alain{???}s'il existe une mot $u\in A^*$ cyclable à partir de $\vect{x}$ tel que $\valeur{u}(i) > 0$ pour tout $i\in J$ %J= truc infini

%\begin{Theorem} % FAUX !!!
%Soit $S = (A,\lambda)$ un VAS et $\vect{x},\vect{y}\in\Nomega^d$ deux $\omega$-configurations.
%$\vect{y}$ est générable en dessous de $\vect{x}$ 
%ssi il existe deux configurations finies $\vect{x'},\vect{y'}\in\N^d$ et un mot $u\in A^*$ tels que
%\begin{enumerate}
%    \item $\vect{x'} \in\; \downarrow\vect{x}$ ($\vect{x'}$ est dans l'idéal représenté par $\vect{x}$)
%    \item $\vect{y'} \eqfin \vect{y}$ (égalité des coordonnées finies)
%    \item $\vect{x'} \trans{}{S} \vect{y'}$
%    \item $u$ est cyclable à partir de $\vect{y}$
%    \item $\valeur{u}(i) > 0$ pour tout $i\in \Jinf{\vect{x}}$
%\end{enumerate}
%\end{Theorem}

\begin{Theorem}\alain{ce théorème n'a rien à faire là}
\lucas{À rectifier}
Soient $S = (A,\lambda)$ un VAS et $\vect{x}\in\Nomega^d$ \xout{une $\omega$-configuration}.
On a $\downarrow \vect{x} \inter \overline{\inirat(S)} \neq \vide$ ssi
il existe deux mots $u,v \in A^*$ vérifiant
\begin{enumerate}
    \item $\valeur{u} \geq \vect{0}$,
    \item Pour tout $i \in I$, $\valeur{u}(i)=0$ implique $\valeur{v}(i) \geq 0$ ,
    \item Il existe $i\in I$ tel que $\valeur{v}(i) < 0$,
    \item On peut accéder \alain{vague} depuis $\vect{x}$ à une $\omega$-configuration $\vect{y}$ à partir de laquelle $u$ est franchissable.
\end{enumerate}

\lucas{On se restreint aux coordonnées de $\Jfin{\vect{x}}$ et on demande à surpasser la configuration obtenue en prenant la valeur minimale pour chaque coordonnée franchissant $u$.}
\end{Theorem}

%*****

\begin{proof}
\lucas{Pas encore au point, ne pas regarder}

Supposons que $\downarrow \vect{x} \inter \overline{\inirat(S)} \neq \vide$.
On a donc une configuration initiale $\xinit \in \N^d$ telle que $\lang(S,\xinit)$ n'est pas rationnel.
Il existe alors (d'après le théorème \ref{caractérisation}) pour tout $k\in\N$ une séquence d'actions franchissable 
$\xinit \trans{u_k}{S} \vect{y}_k \trans{v_k}{S} \vect{z}_k$
et une coordonnée $i_k \in I$ tels que $\valeur{v_k}(i_k) \leq k$.

Remarquons qu'une séquence d'action qui convient pour un certain $k\in\N$ fonctionne aussi pour les valeurs de $k$ inférieures.
Par conséquent, on peut sélectionner uniquement les séquences qui nous intéresse, pour peu qu'il  en reste une infinité :

\begin{itemize}
    \item Comme $I$ est fini, les indices $i_k$ ne peuvent prendre qu'un nombre fini de valeurs possibles, donc (par le principe des tiroirs) il existe un $\mathrm{i}\in I$ apparaissant une infinité de fois.
    On peut ainsi supposer que $i_k = \mathrm{i}$ pour tout $k$.
    %\alain{expliquer même si ça semble évident}
    
    \item Comme $(\N^d,\leq)$ forme un bel ordre, la suite $(\valeur{u_k})_{k\in\N}$ admet une sous-suite croissante pour $\leq$.
    On peut donc se ramener au cas où $(\valeur{u_k})_{k\in\N}$ est croissante.
\end{itemize}

Notons $J\subseteq I$ l'ensemble des coordonnées bornées dans les $\valeur{u_k}$.
Formellement, on définit 
$$J = \set{j\in I}{sup \set{\valeur{u_k}(j)}{k\in\N} \text{ est fini}}
\text{, et pour } j\in J,\hspace{2mm} h_j = \max \set{\valeur{u_k}(j)}{k\in\N}$$

Il nous faut maintenant définir des mots $u$ et $v$ vérifiant les trois points du théorème.
Commençons par fabriquer $u$ satisfaisant le point 1 :
\begin{itemize}
    \item Pour que les coordonnées dans $J$ des $\valeur{u_k}$ soient positives pour chaque $k\in\N$, 
    on construit $u_k'$ en retirant de $u_k$ pour tout $j\in J$ les $h_j$ premières actions qui font décroître la coordonnée $j$ (ou moins s'il n'y en a plus).
    On obtient ainsi $\valeur{u_k'}(j) \geq 0$ pour tout $k\in\N$ et $j\in J$.
    
    \item Posons $\xi = \set{\abs{\valeur{a}(i)}} {a\in A,i\in I}$.
    Lors du passage de $u_k$ à $u_k'$, les coordonnées $i\in I\setminus J$ ne peuvent décroître d'au plus $\xi \times \sum_{j\in J} h_j$.
    Par ailleurs, comme $(\valeur{u_k})_{k\in\N}$ est croissante et que pour tout $i\in I\setminus J$, la suite d'entiers $(\valeur{u_k}(i))_{k\in\N}$ tend vers l'infini, 
    il existe un $k_0\in\N$ tel que pour tout $i\in I\setminus J$, $\valeur{u_{k_0}}(i) > \xi \times \sum_{j\in J} h_j$.
    On en déduit que $\forall i\in I\setminus J, \hspace{1mm} \valeur{u_{k_0}}(i) > 0$.
\end{itemize}
Par conséquent, le mot $u_{k_0}'$ vérifie le point 1 %du théorème \ref{conf_struct_rat}
: $\valeur{u_{k_0}'} > \vect{0}$.
\vspace{2mm}

Maintenant, procédons de façon similaire sur les $v_k$.
Pour tout $j\in J$, $(\valeur{u_k}(j))_{k\in\N}$ est borné, donc $(\vect{x}_k(j))_{k\in\N}$ est borné également par $\ell_j = h_j + \xinit(j)$.
On construit alors $v_k'$ en retirant de $v_k$ pour tout $j\in J$ les $\ell_j$ premières actions qui font décroître la coordonnée $j$.
Tous les $v_k'$ vérifient le point 2 (avec $u = u_{k_0}'$), puisque si pour un $i\in I$, $\valeur{u_{k_0}'}(i) = 0$, alors $i\in J$, d'où $\valeur{v_k'}(i) \geq 0$.

Enfin, en prenant $k_1 = 1 + \xi \times \sum_{j\in J} \ell_j$, on a $\valeur{v_{k_1}}(\mathrm{i}) \leq k_1$, d'où $\valeur{v_{k_1}'}(\mathrm{i}) < 0$, et $v_{k_1}'$ vérifie aussi le point 3.
Finalement, $u_{k_0}'$ et $v_{k_1}'$ vérifient les trois points %du théorème \ref{conf_struct_rat}
, ce qui montre l'implication directe.
\vspace{5mm}

Réciproquement, supposons qu'il existe deux mots $u,v\in A^*$ vérifiant les points 1, 2, et 3.
Soit $\xinit$ une configuration à partir de laquelle $u$ est franchissable (on choisit des coordonnées aussi grandes que nécessaire).
Montrons que $\lang(S,\xinit)$ n'est pas rationnel.

Soit $k\in\N$.
Notons $\rho = \text{max}\set{-\valeur{v}(i)}{i\in I}$, 
et montrons alors que $u^{\rho\times k} v^k \in \lang(S,\xinit)$.

\begin{itemize}
    \item D'après le point 1, on a $\valeur{u} > \vect{0}$, donc $\xinit + \valeur{u} > \xinit$.
    Comme $u$ est franchissable à partir de $\xinit$, le mot $u^{\rho\times k}$ l'est également.
    Posons $\vect{x}_k = \xinit + \valeur{u^{\rho\times k}}$ la configuration atteinte.
    
    \item D'après le point 2, si pour un $i\in I$, on a $\valeur{v^k}(i) < 0$, alors $\valeur{u}(i) > 0$, d'où $\vect{x}_k(i) \geq \rho\times k$.
    Par définition de $\rho$, on en déduit que $\valeur{v^k}$ a bien les ressources nécessaires pour se déclencher à partir de $\vect{x}_k$, donc est bien franchissable.
    Posons $\vect{y}_k = \vect{x}_k + \valeur{v^k}$ la nouvelle configuration atteinte.
    
    \item Enfin, d'après le point 3, il existe un $i\in I$ tel que $\valeur{v}(i) < 0$, d'où $\valeur{v^k}(i) \leq -k$.
\end{itemize}
Par conséquent, l'exécution $\xinit \trans{u^{\rho\times k}}{S} \vect{x}_k \trans{v^k}{S} \vect{y}_k$ vérifie la propriété \eqref{eq:caracterisation} pour ce $k$.
On notera que $\xinit$ est bien indépendant de $k$.
Le langage $\lang(S,\xinit)$ n'est donc pas rationnel, et $S$ n'est pas structurellement rationnel.
\end{proof}

%==================================================

\section{Calcul des configurations maximales rationnelles}
\alain{on fait plus puisqu'on calcule les éléments minimaux pour lesquels le VAS n'est pas rationnel}

\begin{Definition}
Pour un VAS $S$, une configuration $\vect{x} \in \N^d$ est \emph{rationnelle pour $S$} si $\lang(S,\vect{x})$ est rationnel.
\end{Definition}

Nous allons considérer l'ensemble des configurations $\vect{x}$ pour lesquelles ${\lang(S,\vect{x}) \text{ est rationnel }}$ et son complémentaire, l'ensemble des configurations $\vect{x}$ pour lesquelles ${\lang(S,\vect{x}) \text{n'est pas rationnel}}$. 
%	Comme  ainsi que l'ensemble de ses éléments minimaux.
%
	Dans cette partie, le comportement d'un VAS est donc étudié à partir de l'ensemble de ses configurations initiales possibles c'est-à-dire pour tout les $\vect{x} \in \N^d$.
%
Posons donc: $$\inirat(S) \defeq \set{\vect{x} \in \N^d} {\lang(S,\vect{x}) \text{ est rationnel}}$$

\noindent
Rappelons que nous notons $\overline{\inirat(S)}
%	\set{\vect{x} \in \N^d} {\lang(S,\vect{x}) \text{ n'est pas rationnel}}  
    = \N^d - \inirat(S)$.
%
Montrons que si $\lang(S,\vect{x})$ n'est pas rationnel alors $\lang(S,\vect{x'})$ n'est pas rationnel pour tout $\vect{x'} \geq \vect{x}$ ce qui revient à dire que la propriété de non-rationalité est monotone.
%\xout{selon les configurations initiales choisies.}

\begin{Proposition}\label{monotonie_rationnel}
Pour tout VAS $S = (A,\lambda)$, on a
%	l'ensemble $\overline{\inirat(S)}$ est clos par le haut 
 $\overline{\inirat(S)} = \uparrow \Min(\overline{\inirat(S)})$ 
%	où $\Min(\overline{\inirat(S)})$ %	est l'ensemble fini des éléments minimaux de $\overline{\inirat(S)}$.
\end{Proposition}

%\alain{peut-on déduire que $\set{\vect{x}}{\lang(S,\vect{x}) \text{ est  rationnel}}$ est clos par le bas ?}
%\lucas{Oui, c'est le complémentaire.}

\begin{proof}
Montrons d'abord que $\overline{\inirat(S)}$ est clos par le haut. Soit $\vect{x},\vect{x'}\in\N^d$ deux configurations vérifiant $\vect{x} \leq \vect{x'}$.
Supposons que $\lang(S,\vect{x})$ ne soit pas rationnel.

D'après le théorème \ref{caractérisation}, il existe pour tout $k\in\N$ une exécution $\vect{x} \trans{u}{S} \vect{y}_k \trans{v}{S} \vect{z}_k$ vérifiant $\valeur{v} = \vect{z}_k - \vect{y}_k \not\geq -k$.
Comme $\vect{x}\leq\vect{x'}$, du fait de la monotonie des VAS, la séquence $uv$ est aussi franchissable à partir de $\vect{x'}$, et l'on obtient l'exécution $\vect{x'} \trans{u}{S} \vect{y'}_k \trans{v}{S} \vect{z'}_k$.
où $\vect{y'}_k = \vect{y}_k + (\vect{x'} - \vect{x})$ et $\vect{z'}_k = \vect{z}_k + (\vect{x'} - \vect{x})$.

On a alors $\vect{z'}_k - \vect{y'}_k = \valeur{v} \not\geq -k$, donc $(S,\vect{x'})$ ne satisfait pas non plus la propriété $\eqref{eq:caracterisation}$.
On en déduit que $\lang(S,\vect{x'})$ n'est pas rationnel.

L'ensemble $\Min(\overline{\inirat(S)})$ est fini car $\leq$ est un bel-ordre sur $\N^d$. On déduit que $\overline{\inirat(S)}=\uparrow \Min(\overline{\inirat(S)})$ d'après la proposition \ref{clos_haut_wqo}.
\end{proof}

\begin{Remark}
On en déduit que $\inirat(S) = \set{\vect{x} \in \N^d} {\lang(S,\vect{x}) \text{ est rationnel}}$ est clos par le bas. Nous allons calculer une représentation finie de $\inirat(S)$ en calculant $\Min(\overline{\inirat(S)})$ qui est une représentation finie de son complémentaire $\overline{\inirat(S)}$.
\end{Remark}

Montrons maintenant que l'ensemble fini $\Min(\overline{\inirat(S)})$ est calculable.
On utilise pour cela sur un résultat de Valk \& Jantzen \cite{vaja85}.

\begin{Theorem}[\cite{vaja85} théorème 2.14]
Soit $K \subseteq \N^d$ un ensemble clos par le haut.
Alors $\Min(K)$ est calculable si et seulement si pour tout $\vect{x} \in \Nomega^d$, le prédicat $p_K(\vect{x}) = (\downarrow \vect{x} \cap K \neq \vide)$ est décidable.
\end{Theorem}

\begin{Proposition}\label{minimaux}
L'ensemble $\Min(\overline{\inirat(S)})$ est fini et calculable pour tout VAS $S$.
\end{Proposition}

\begin{proof}
Si $\vect{x}\in\N^d$, le prédicat $p_{\overline{\inirat(S)}}(\vect{x})$ est équivalent au prédicat "$\lang(S,\vect{x})$ est rationnel".
Reste à considérer les cas où certaines coordonnées de $\vect{x}$ valent $\omega$.

Montrons que le prédicat $p_K(\vect{x})=(\downarrow \vect{x} \cap \N^d \cap K \neq \emptyset)$ est décidable pour $K=\overline{\inirat(S)}$. Le principe doit être le même que la preuve du théorème 3.11 (toujours dans Valk) pour les 4 ensembles de marquages étudiés.
\alain{la preuve est à faire !!! }
\end{proof}

%\begin{Corollary}\label{minimaux_cor}
%L'ensemble $\Min(\overline{\inirat(S)})$ est fini et calculable pour tout VAS $S$.
%\end{Corollary}

%On en déduit le corollaire suivant :

\begin{Proposition}\label{maximaux}
L'ensemble des configurations rationnelles est clos par le bas et son ensemble d'idéaux maximaux est fini et calculable pour tout VAS $S$ 
\end{Proposition}

\begin{proof}
On utilise le fait qu'on peut passer d'une représentation finie d'un ensemble clos par le haut dans $\N^d$ à une représentation finie de son complémentaire (qui est clos par le bas) dans $\N^d$ \cite{GHKNS-til2020}.

%@incollection{GHKNS-til2020,
%  volume = 53,
%  series = {Trends In Logic},
%  publisher = {Springer},
%  booktitle = {Well-Quasi Orders in Computation, Logic, Language and Reasoning},
%  editor = {Schuster, Peter M. and Seisenberger, Monika and Weiermann, Andreas},
%  author = {Jean Goubault{-}Larrecq and Simon Halfon and P. Karandikar and K. {Narayan Kumar} and {\relax Ph}ilippe Schnoebelen},
%  title = {The Ideal Approach to Computing Closed Subsets in Well-Quasi-Orderings},
%  pages = {55-105},
%  year = 2020,
%  doi = {10.1007/978-3-030-30229-0_3}
%}
\end{proof}

Une conséquence de la section xxx permet de conclure que le problème est décidable puisque cela revient à calculer $\Min(\overline{\inirat(S)})$ et à décider si $\Min(\overline{\inirat(S)})=\vide$.

\begin{Remark}
Pour tout VAS $S$, nous avons donc découpé l'ensemble $\N^d$ des configurations initiales en deux sous-ensembles disjoints, $\inirat(S)$ et son complémentaire $\overline{\inirat(S)}$, tels que $\inirat(S)$ est clos par le bas (et $\overline{\inirat(S)}$ est clos par le haut).
Quelle est la complexité du calcul de $\Min(\overline{\inirat(S)})$ ? L'algorithme proposé demande de calculer $\clover(S)$ donc la complexité de l'algorithme est Ackermann. Ceci ne veut pas dire qu'on ne puisse pas faire mieux. 
\end{Remark}

%----------------------------------------------------

\section{Commentaires}
Vérifier qu'on peut énoncer Vidal-Naquet sur le graphe de couverture minimal défini par le graphe de Karp-Miller dans lequel on a gardé que les marquages maximaux.\\
Vérifier que ce nouveau graphe peut être obtenu à partir de Clover en ajoutant les transitions possibles (prolongées par continuité sur $\N^d$). Vérifier qu'il ne manque pas de transitions utiles.

Tenter de se débarrasser du graphe, de Clover, voire plus dans la preuve de Valk et Vidal-Naquet.

Simplifier le thm 12 pour avoir un calcul facile dans les cas faciles (borné).

Voir Garey, Johnson : référence pour les problèmes de complexité sur les vecteurs

Regarder taille de l'automate (minimal ?) construit par GY et VVN. Est-ce Ackermann ?

Réduire la rationalité à la bornitude, la terminaison ou la couverture.

Programme:
recherche jusqu'au 15 juin
écriture article du 15 juin au 15 juillet \\

\section{Conclusion et perspectives}
que sait-on décider et à quel coût pour les regular VASS ?

étudier le model checking pour  $CTL$  et $CTL^*$ des regular VASS.
\begin{itemize}
  \item On remarque qu'on peut décider la rationalité avec Clover et qu'on n'a pas besoin du graphe de couverture ni même de l'ensemble des configurations de ce graphe. Cet énoncé revient aussi à dire qu'un certain nombre de VAS associés à $S$ sont bornés, ce qui a comme conséquence que la donnée de Clover est suffisante pour décider la rationalité dun VAS. La preuve de VV nécessite la construction du graphe de couverture dont la taille est au pire Ackermann tandis que celle de yoeli n'utilise que l'ensemble des configurations du graphe de Karp-Miller; cela dit cet ensemble peut aussi être de taille Ackermann. Mais parfois l'ensemble des éléments maximaux de l'ensemble des configurations du graphe de Karp-Miller, appelé Clover, est petit  et facile à calculer d'un point de vue algorithmique.  
\item appliquer aux affine VAS et aux very-WSTS.
\end{itemize}

\section{Appendix}


\subsection{La relation d'équivalence de Ginzburg et Yoeli n'est pas d'index fini, Counter-example for Lemme 2 dans \cite{giyo80}}

Ginzburg et Yoeli introduisent dans \cite{giyo80} une relation d'équivalence $\relGY$ sur les configurations et énoncent que $\lang(S)$ est rationnel si et seulement si $\relGY$ admet un nombre fini de classes d'équivalence dans $\reach(S)$ (\cite{giyo80}, Théorème 1).

S'il est vrai que $\reach(S)/\relGY$ fini implique que $\lang(S)$ est rationnel, la réciproque est fausse et nous donnerons un contre-exemple d'un langage $\lang(S)$ rationnel tel que $\reach(S)/\relGY$ est infini. 
Nous proposerons de reprendre l'idée de Ginzburg et Yoeli,
mais en définissant une autre relation d'équivalence pour laquelle on obtiendra cette fois-ci l'équivalence entre la rationalité du langage et le quotient fini selon cette relation.

\begin{Definition}[\cite{giyo80} section 3]
Soit $S=(A,\lambda,\xinit)$ un VAS. La relation $\relGY$ est définie pour tout $\vect{x},\vect{y} \in\reach(S)$ par : 
$$\vect{x}\relGY\vect{y} \ssi \forall w\in A^\ast, \big( \vect{x} +\valeur{w}\in\reach(S) \equivaut \vect{y} +\valeur{w}\in\reach(S) \big)$$
\end{Definition}
\alain{ça donne quoi si on définit $\relGY$ sur tout $\N^d$ ? Une classe de plus seulement ? Une infinité ?}

\begin{Remark}
$\relGY$ est une relation d'équivalence sur l'ensemble $\reach(S)$ des configurations accessibles.
\end{Remark}

On aurait envie d'obtenir un résultat similaire à celui de Nérode, à savoir dire que $\lang(S)$ est rationnel si et seulement si $\relGY$ admet un nombre fini de classes d'équivalence.
Cela est malheureusement faux, puisque pour $\vect{x}\in\reach(S)$ et $w\in A^\ast$, l'écriture $\vect{x} +\valeur{w}\in\reach(S)$ ne permet pas de dire si la séquence $w$ est franchissable à partir de $\vect{x}$.
Il pourrait en effet exister une autre séquence $w'\in A^\ast$ franchissable à partir de $\vect{x}$ aboutissant à la configuration $\vect{x} +\valeur{w'} = \vect{x} +\valeur{w}$,
voire même un moyen d'accéder à la configuration $\vect{x} +\valeur{w} = \xinit +\valeur{u}$ depuis la configuration initiale par une autre séquence d'action $u\in A^\ast$ sans que $\xinit +\valeur{u}$ ne soit accessible depuis $\vect{x}$.

\alain{énoncer le lemma faux de Yoeli et expliquer et donner le contre-exemple}
Plus précisément sur la preuve de \alain{non l'énoncé } de Ginzburg et Yoeli, 
avoir $\reach(S)/\relGY$ fini implique bien $\lang(S)$ rationnel, ce qui est prouvé en construisant explicitement l'automate.
Par contre, la réciproque est fausse : 
L'erreur (avant-dernière ligne de la preuve du théorème 1 de \cite{giyo80}) était d'affirmer que savoir $\xinit +\valeur{uw}\in\reach(S)$ pour $u\in\lang(S)$ et $w\in A^\ast$ permettait d'en déduire que $uw\in\lang(S)$.

\vspace{5mm}

On donne ci-dessous un contre-exemple pour illustrer ce point.
Il est nécessaire de se placer au moins en dimension 3, car le résultat de Ginzburg et Yoeli reste vrai en dimension inférieure.

\lucas{Ajouter preuve que le résultat reste vrai en dimension inférieure à 2.}

\begin{Example}
Soit le $3-$VAS $S = (A=\os{a,b,c}, \lambda, \xinit=(0,0,0))$ dont les actions sont étiquetés par $\valeur{a}=(1,0,0)$, $\valeur{b}=(0,1,-1)$ et $\valeur{c}=(-1,-1,1)$.
Le langage reconnu $\lang(S)=a^\ast$ est rationnel, et les configurations accessibles sont les $\vect{x}_n=(n,0,0)$ pour $n\in\N$.

Cependant, pour deux entiers $m>n>0$, bien que $\lang(S,\vect{x}_m) =\lang(S,\vect{x}_n) =\lang(S)$, on a $\vect{x}_m \not\relGY \vect{x}_n$ :
Cela se constate en considérant la séquence d'actions $b^{n+1}c^{n+1}$ qui n'est jamais franchissable, mais qui vérifie $\vect{x}_m +\valeur{b^{n+1}c^{n+1}} = (m-n-1,0,0)\in \reach(S)$ alors que $\vect{x}_n +\valeur{b^{n+1}c^{n+1}} = (-1,0,0)\notin \reach(S)$.

La relation $\relGY$ admet alors une infinité de classes d'équivalences $(\os{\vect{x}_n})_{n\in\N}$ sur $S$.
\end{Example}


\section{FSTTCS 2021}  
41st IARCS Annual Conference on
Foundations of Software Technology and Theoretical Computer Science

    FSTTCS 2021: December 15–18, 2021.
    Post-conference workshops: December 19, 2021.

deadline: unknown but usually 15-25 july, say {\bf 15 july}

Submissions must be in electronic form via EasyChair using the LIPIcs LaTeX style file. Submissions must not exceed 12 pages (excluding bibliography), but may include a clearly marked appendix containing technical details. The appendix will be read only at the discretion of the program committee. Simultaneous submissions to journals or other conferences with published proceedings are disallowed. 

%==============================================

\bibliographystyle{plain}
\bibliography{biblio}

\end{document}