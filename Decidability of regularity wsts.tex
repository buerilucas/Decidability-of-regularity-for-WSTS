\documentclass[a4paper,final]{article}

\usepackage[utf8]{inputenc}
\usepackage[T1]{fontenc}
\usepackage[french]{babel}

\usepackage{xcolor}
\usepackage{enumitem}
\usepackage{xspace}
\usepackage{amsmath}
\usepackage{amssymb}
\usepackage{amsthm}
\usepackage{array}
\usepackage[french]{algorithm2e}
\usepackage{stmaryrd}

\usepackage{geometry}
\geometry{hmargin=3cm,vmargin=2cm}

%\theoremstyle{definition}
%\newtheorem{Question}{Question}

\let\oldphi\phi
\let\phi\varphi
\let\oldepsilon\epsilon
\let\epsilon\varepsilon
\let\leq\leqslant
\let\geq\geqslant

%================================================================

\newcommand{\set}[2]{\left\{#1\mathrel{\left|\vphantom{#1}\vphantom{#2}\right.}#2\right\}}
\newcommand{\os}[1]{\left\{\mathinner{#1}\right\}}
\newcommand{\defeq}{\ensuremath{\stackrel{\textit{def}}{=}}}
\let\union\cup
\let\inter\cap

\newcommand{\N}{\ensuremath{\mathbb{N}}}

\newcommand{\petri}{réseau de Petri\xspace}
\newcommand{\fire}[2]{\ensuremath{#1 (#2\rangle}}
\newcommand{\lang}{\ensuremath{\mathcal{L}}}

%================================================================

\title{Décidabilité de la rationnalité pour les WSTS}
\author{Lucas \textsc{Bueri}}
\date{Stage M2 - 2021}

\begin{document}

\maketitle

%================================================================

\section{Réseaux de Petri}

Un \petri $N = (P,T,B,F,M_0)$ est la donnée de
\begin{itemize}
    \item un ensemble fini $P$ de $r$ emplacements,
    \item un ensemble fini $T$ de $s$ transitions,
    \item une fonction de coût $B: P\times T\to\N$,
    \item une fonction de production $F: P\times T\to\N$,
    \item un marquage initial $M_0: P\to\N$.
\end{itemize}

Les configurations sont les marquages $M: P\to\N$, aussi considérés comme les valeurs possibles de $r$ compteurs (vecteur de $\N^r$).
On peut déclencher la transition $t$ à partir du marquage $M$ si et seulement si $M(p)\geq B(p,t)$ pour tout $p\in P$ (noté $M\geq B(\cdot,t)$).

On obtient alors un nouveau marquage $M'$ défini par $M':=M+D(\cdot,t)$ où $D\defeq F-B$. 
$B$ représente donc le coût de la transition (le nombre de jetons requis et consommés dans chaque emplacement), et $F$ représente sa production (les jetons créés lors du déclenchement).

On notera $\fire{M}{t}$ lorsque $t$ peut se déclencher sur $M$, et $\fire{M}{t}M'$ si déclencher $t$ sur $M$ donne $M'$.
On étendra naturellement cette notation (ainsi que $B(p,\cdot)$ et $F(p,\cdot)$) aux séquences de transitions, ou mots $w\in T^\ast$.

On s'intéresse aux exécutions possibles de par deux manières différentes :
\begin{enumerate}
    \item D'abord en regardant le langage $\lang(N)\defeq \set{w\in T^\ast} {\fire{M_0}{w}}$ des séquences de déclenchement autorisés depuis le marquage initial, éventuellement en n'autorisant que quelques marquages finals, ou en omettant certaines transitions au travers d'un étiquetage des transitions.
    \item Ensuite en regardant les marquages accessibles depuis $M_0$ en utilisant une séquence quelconque de transitions.
\end{enumerate}

Pour cela, on considère le graphe de couverture $CG(N)$ du \petri.

%----------------------------------------------------------------

\section{Graphe de couverture}

On ajoute une valeur $\omega$, symbolisant une quantité aussi grande que nécessaire de jetons dans un emplacement, et on pose $\N_\omega\defeq \N\union\os{\omega}$.

On commence alors par étendre la notion de marquage aux $Q:P\to\N_\omega$ ordonnés par $\leq$ en posant $Q\leq Q'$ lorsque pour tout $p\in P$, $Q'(p)=\omega$, ou $Q(p)$ et $Q'(p)$ sont deux entiers vérifiant $Q(p)\leq Q'(p)$.
Les transitions s'appliquent également à ces nouveaux marquages :
la transition $t$ se déclenche sur $Q$ uniquement lorsque $Q\geq B(\cdot,t)$, et aboutit au marquage $Q'$ définit par $Q':=Q+D(\cdot,t)$ où $D=F-B$ et où le $+$ est étendu à $\N_\omega$ en posant $\omega+\omega=\omega+n=n+\omega=\omega$.

On donne une représentation graphique des exécutions possibles.
Les noeuds ou sommets sont étiquetés par des marquages de $\N_\omega^r$, et les arêtes par des transitions de $T$.

On commence par définir l'arbre de couverture $CT(N)$ en partant du marquage initial $r_0:M_0$ comme racine, et on construit l'arbre par induction structurelle.
Pour chaque noeud de l'arbre $s$ associé à un marquage $Q:P\to\N_\omega$ :
\begin{itemize}
    \item Si $s$ est précédé par un noeud $r:Q'$ étiqueté par le même marquage $Q'=Q$, alors $s$ est une feuille.
    Nul besoin d'indiquer l'action des transitions sur $r$ puisque cela a déjà été fait sur $s$.
    \item Si $s$ est précédé par un noeud $r:Q'$ vérifiant $Q'>Q$, alors on remplace $Q$ par $\tilde{Q}$ pour l'étiquette de $s$ en posant $\tilde{Q}(p):=Q(p)$ pour les $p\in P$ tels que $Q(p)=Q'(p)$, et $\tilde{Q}(p)=\omega$ si $Q(p)>Q'(p)$.
\end{itemize}

%================================================================

\end{document}
