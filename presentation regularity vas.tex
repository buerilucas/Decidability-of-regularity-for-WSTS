\documentclass[french]{beamer}
\usepackage[utf8]{inputenc}

\usepackage[T1]{fontenc}
\usepackage{babel}

\usepackage{xcolor}
%\usepackage{enumitem}
\usepackage{xspace}
\usepackage{amsmath, amssymb}
\usepackage{amsthm}
%\usepackage{graphicx}

%\usepackage{tikz} %to draw automata
%\usetikzlibrary{automata, positioning, arrows}

%\setbeamertemplate{footline}[frame number]
\usetheme{Warsaw}
\expandafter\def\expandafter\insertshorttitle\expandafter{%
  \insertshorttitle\hfill%
  \insertframenumber\,/\,\inserttotalframenumber}
\beamertemplatenavigationsymbolsempty
%\setbeamertemplate{footline}[frame number]

\newcommand{\set}[2]{\left\{#1\mathrel{\left|\vphantom{#1}\vphantom{#2}\right.}#2\right\}}
\newcommand{\os}[1]{\left\{\mathinner{#1}\right\}}
\newcommand{\defeq}{\ensuremath{\stackrel{\textit{def}}{=}}}
\newcommand{\eqfin}{\ensuremath{=_\text{fin}}}
\let\union\cup
\let\inter\cap
\newcommand{\abs}[1]{\ensuremath{\displaystyle\left\lvert #1 \right\rvert}}
\let\vide\varnothing
\newcommand{\Min}{\textit{Min}}
\newcommand{\Max}{\textit{Max}}

\newcommand{\N}{\ensuremath{\mathbb{N}}}
\newcommand{\Z}{\ensuremath{\mathbb{Z}}}
\newcommand{\Nomega}{\ensuremath{\mathbb{N}_\omega}}
\newcommand{\indicatrice}[1]{\ensuremath{\mathds{1}_{#1}}}
\newcommand{\card}[1]{|#1|}

\newcommand{\petri}{réseau de Petri\xspace}
\newcommand{\fire}[2]{\ensuremath{#1 (#2\rangle}}
\newcommand{\lang}{\ensuremath{\mathcal{L}}}
\newcommand{\reach}{\ensuremath{\textit{Reach}}}
\newcommand{\cover}{\ensuremath{\mathcal{C}}}
\newcommand{\clover}{\textit{Clover}\xspace}
\newcommand{\trans}[2]{\ensuremath{\stackrel{#1}{\longrightarrow}_{#2}}}
\newcommand{\transZ}[1]{\ensuremath{\stackrel{#1}{\rightsquigarrow}}}

\newcommand{\vect}[1]{\ensuremath{\mathbf{#1}}}
%\newcommand{\action}[1]{\ensuremath{\mathbf{#1}}}
\newcommand{\rel}{\ensuremath{\equiv}}
\newcommand{\relGY}{\ensuremath{\equiv^\text{GY}_S}}
\newcommand{\ssi}{\ensuremath{\text{ ssi }}}
\newcommand{\equivaut}{\ensuremath{\Leftrightarrow}}
\newcommand{\xinit}{\ensuremath{\vect{x}_\text{init}}}
\newcommand{\valeur}[1]{\ensuremath{\overline{#1}}}
\newcommand{\inirat}{\mathcal{R}}
\newcommand{\unite}[1]{\ensuremath{\vect{u}_{#1}}}

\newcommand{\Jfin}[1]{J^\text{fin}_{#1}}
\newcommand{\Jinf}[1]{J^\text{inf}_{#1}}

\newtheorem{proposition}{Proposition}
\newtheorem{syntax}{Syntax}
\newtheorem{semantic}{Semantic}
\newtheorem{summary}{Summary}

\let\oldemph\emph
\renewcommand{\emph}[1]{\oldemph{\color{blue}#1}}

%%%%%%%%%%%%%%%%%%%%%%%%%%%%%%%%%%%%%%%%%%%%%%%%%%%

\title[Decidability of Regularity for VAS]{Décidabilité de la Rationalité pour les VAS}
\author[Lucas Bueri - avec Alain Finkel (LMF)]{\textbf{Lucas Bueri}}
\institute{\normalsize Stage de recherche de M2 - du 22 mars au 30 juillet 2021\\
\vspace{10pt}
avec Alain Finkel\\
\textit{LMF, Université Paris-Saclay}}
\date{\textit{MPRI} \\ Jeudi 2 septembre 2021}

\begin{document}
 \maketitle
 
\begin{frame}
\frametitle{L'environnement de recherche}

\end{frame}


%%%%%%%%%%%%%%%%%%%%%%%%%%%%%%%%%%%%%%%%%%%%%%%%%%

\begin{frame}{Système d'Addition de Vecteurs}
\begin{definition}[Syntaxe]
Un $d-$VAS $S=(A,\lambda)$ est composé
\begin{itemize}
    \item d'un alphabet fini $A$ d'actions,
    \item d'un étiquetage $\lambda:A\to\Z^d$ injectif.
    On notera $\valeur{a}=\lambda(a)$.
\end{itemize}
\end{definition}

\begin{definition}[Sémantique]
Les \emph{configurations} de $S$ sont les vecteurs de $\N^d$.

L'action $a\in A$ est \emph{franchissable} 
à partir de $\vect{x}\in \N^d$ lorsque $\vect{x} + \valeur{a} \geq \vect{0}$.
Son franchissement donne $\vect{x}\trans{a}{S} \vect{y} = \vect{x} + \valeur{a}$.
\end{definition}
\end{frame}

\begin{frame}{Exemple d'un VAS}
\begin{definition}[Langage]
\begin{itemize}
    \item 
    $\lang(S,\vect{x})\defeq \set{w\in A^\ast} {\exists \vect{y}\in\N^d, \vect{x}\trans{w}{S}\vect{y}}$ est le langage des séquences d'actions franchissables à partir de $\vect{x}$,
    
    \item $\reach(S,\vect{x})\defeq \set{\vect{y}\in\N^d} {\exists w\in A^\ast, \vect{x}\trans{w}{S} \vect{y}}$ est l'ensemble des configurations accessibles à partir de $\vect{x}$. 
\end{itemize}
\end{definition}

\vspace{3mm}
Exemple : 
Soit $S = (\os{a,b,c}, \lambda, \xinit=(0,0,0))$ 

%dont les actions sont étiquetés par 
avec $\valeur{a}=(1,0,0)$, $\valeur{b}=(0,1,-1)$ et $\valeur{c}=(-1,-1,1)$.

\vspace{3mm}
$\lang(S)= a^\ast$ est rationnel, et $\reach(S)= \set{\vect{x}_n= (n,0,0)} {n\in\N}$.
\end{frame}


\begin{frame}{Clover et le graphe de couverture}
    
\end{frame}

\begin{frame}{Une relation d'équivalence sur $\N^d$}
\begin{definition}[Relation sur les configurations de $S$] 
$\forall \vect{x},\vect{y}\in\N^d, \vect{x} \rel_S \vect{y} \ssi \lang(S,\vect{x}) = \lang(S,\vect{y})$
\end{definition}

\begin{definition}[Relation de Ginzburg et Yoeli] 
$\vect{x}\relGY\vect{y} \ssi \forall w\in A^\ast, \big( \vect{x} +\valeur{w}\in\reach(S) \equivaut \vect{y} +\valeur{w}\in\reach(S) \big)$
\end{definition}

\begin{proposition}
Si $\vect{x} \relGY \vect{y}$ alors $\vect{x} \rel_S \vect{y}$.
Mais $\rel_S \not\subseteq \relGY$.
\end{proposition}

Contre-exemple : 
si $m>n>0$, on a 
\begin{itemize}
    \item $\lang(S,\vect{x}_m) =\lang(S,\vect{x}_n) =\lang(S)$,
    \item mais $\vect{x}_m \not\relGY \vect{x}_n$ (prendre $w= b^{n+1}c^{n+1}$).
\end{itemize}
\end{frame}


\begin{frame}{Caractérisation pour la rationalité}
\begin{theorem}
On a équivalence entre
\begin{enumerate}
    \item $\lang(S)$ est rationnel
    \item $\reach(S)/\rel_S$ est fini
    \item $\exists k\in\N, \forall \vect{x},\vect{y}\in\N^d, 
\big( \xinit\trans{*}{S} \vect{x} \trans{*}{S} \vect{y}\implies
\vect{y}\geq \vect{x} -\vect{k} \big)$
\end{enumerate}
\end{theorem}

\vspace{3mm}
Méthode de Valk et Vidal-Naquet : 
chercher un circuit négatif du graphe de couverture

Méthode corrigée de Ginzburg et Yoeli :
se ramener au problème de bornitude d'un VAS pour chaque élément de Clover. 
%en isolant les coordonnées infinies.
\end{frame}

\end{document}
