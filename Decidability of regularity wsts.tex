\documentclass[a4paper,final]{article}

\usepackage[utf8]{inputenc}
\usepackage[T1]{fontenc}
\usepackage[french]{babel}

\usepackage{xcolor}
\usepackage{enumitem}
\usepackage{xspace}
\usepackage{amsmath}
\usepackage{amssymb}
\usepackage{amsthm}
\usepackage{dsfont}
\usepackage{array}
\usepackage[french]{algorithm2e}
\usepackage{stmaryrd}
\usepackage[normalem]{ulem}

\usepackage{geometry}
\geometry{hmargin=3cm,vmargin=2cm}

\theoremstyle{definition}
\newtheorem{Question}{Question}
\newtheorem{Definition}{Définition}
\newtheorem{Theorem}{Théorème}
\newtheorem{Proposition}{Proposition}
\newtheorem{Lemma}{Lemme}
\newtheorem{Property}{Propriété}

\let\oldphi\phi
\let\phi\varphi
\let\oldepsilon\epsilon
\let\epsilon\varepsilon
\let\leq\leqslant
\let\geq\geqslant

%================================================================

\newcommand{\alain}[1]{\textcolor{blue}{#1}}
\newcommand{\lucas}[1]{\textcolor{purple}{#1}}
\newcommand{\rayer}[1]{\sout{\textcolor{red}{#1}}}


\newcommand{\set}[2]{\left\{#1\mathrel{\left|\vphantom{#1}\vphantom{#2}\right.}#2\right\}}
\newcommand{\os}[1]{\left\{\mathinner{#1}\right\}}
\newcommand{\defeq}{\ensuremath{\stackrel{\textit{def}}{=}}}
\let\union\cup
\let\inter\cap

\newcommand{\N}{\ensuremath{\mathbb{N}}}
\newcommand{\Z}{\ensuremath{\mathbb{Z}}}
\newcommand{\Nomega}{\ensuremath{\mathbb{N}_\omega}}
\newcommand{\indicatrice}[1]{\ensuremath{\mathds{1}_{#1}}}

\newcommand{\petri}{réseau de Petri\xspace}
\newcommand{\fire}[2]{\ensuremath{#1 (#2\rangle}}
\newcommand{\lang}{\ensuremath{\mathcal{L}}}
\newcommand{\trans}[2]{\ensuremath{\stackrel{#1}{\longrightarrow}_{#2}}}
\newcommand{\vect}[1]{\ensuremath{\mathbf{#1}}}
\newcommand{\action}[1]{\ensuremath{\mathbf{#1}}}
\newcommand{\conf}[1]{\ensuremath{\mathcal{R}(#1)}}
\newcommand{\rel}{\ensuremath{E}}
\newcommand{\ssi}{\ensuremath{\text{ ssi }}}
\newcommand{\equivaut}{\ensuremath{\Leftrightarrow}}
\newcommand{\xinit}{\ensuremath{\vect{x}_\text{init}}}
\newcommand{\valeur}[1]{\ensuremath{\overline{#1}}}

%================================================================

\title{Décidabilité de la rationnalité pour les WSTS}
\author{Lucas \textsc{Bueri}}
\date{Stage M2 - 2021}

\begin{document}

\maketitle

%================================================================

\section{Réseaux de Petri}

Un \petri $N = (P,T,B,F,M_0)$ est la donnée de
\begin{itemize}
    \item un ensemble fini $P$ de $d$ emplacements,
    \item un ensemble fini $T$ de transitions,
    \item une fonction de coût $B: P\times T\to\N$,
    \item une fonction de production $F: P\times T\to\N$,
    \item un marquage initial $M_0: P\to\N$.
\end{itemize}

Les configurations sont les marquages $M: P\to\N$, aussi considérés comme les valeurs possibles de $d$ compteurs (vecteur de $\N^d$).
On peut déclencher la transition $t$ à partir du marquage $M$ si et
seulement si $M(p)\geq B(p,t)$ pour tout $p\in P$ (noté $M\geq B(\cdot,t)$).

On obtient alors un nouveau marquage $M'$ défini par $M':=M+D(\cdot,t)$ où $D\defeq F-B$. 
$B$ représente donc le coût de la transition (le nombre de jetons requis et consommés dans chaque emplacement), et $F$ représente sa production (les jetons créés lors du déclenchement).

On notera $\fire{M}{t}$ lorsque $t$ peut se déclencher sur $M$, et $\fire{M}{t}M'$ si déclencher $t$ sur $M$ donne $M'$.
On étendra naturellement cette notation (ainsi que $B(p,\cdot)$ et $F(p,\cdot)$) aux séquences de transitions, ou mots $w\in T^\ast$.

Deux ensembles nous intéresseront alors : le langage $\lang(N)\defeq \set{w\in T^\ast} {\fire{M_0}{w}}$ du \petri et les configurations accessibles $\conf{N}\defeq \set{M':P\to\N}{\exists w\in T^\ast, \fire{M_0}{w}M'}$.

%----------------------------------------------------------------

\section{VAS}

Un \emph{système d'addition de vecteurs de dimension} $d\in\N$ ($d$-VAS) $S=(A,\xinit)$ est la donnée d'un vecteur initial $\xinit\in\N^d$ et d'un ensemble fini $A$ d'\emph{actions}.
À chaque action $a\in A$ est associé un unique vecteur $\valeur{a}\in \Z^d$, de telle manière à ce que deux actions ne soient pas associées au même vecteur de $\Z^d$.
\vspace{3mm}

Les \emph{configurations} de $S$ sont alors les vecteurs de $\N^d$ (à coordonnées positives), 
et chaque action $a\in A$ agit sur $\N^d$ en additionnant à la configuration courante le vecteur $\valeur{a}$ associé.
On a alors une transition entre $\vect{x}$ et $\vect{y}$ étiquetée par l'action $a$ lorsque $\vect{x} + \valeur{a} = \vect{y}$.

De manière équivalente, on dira que l'action $a\in A$ est \emph{franchissable} 
à partir de la configuration $\vect{x}\in \N^d$ lorsque $\vect{x} + \valeur{a} \geq \vect{0}$,
et son déclenchement aboutit à la configuration $\vect{y} := \vect{x} + \valeur{a}$ à travers la transition $(\vect{x},a,\vect{y})\in \N^d\times A\times \N^d$.
On notera $\vect{x}\trans{a}{} \vect{y}$ lorsqu'un tel déclenchement est possible.
\vspace{3mm}

Lorsqu'une séquence d'actions $w= a_1\cdots a_k\in A^\ast$ permet d'aller de $\vect{x}$ à $\vect{y}$ par la séquence de transition $\vect{x}= \vect{x_0}\trans{a_1}{} \vect{x_1}\trans{a_2}{} \dots\trans{a_k}{} \vect{x_k}=\vect{y}$
(où $\vect{x_0},\dots,\vect{x_k} \in\N^d$ et $\vect{x_{i-1}} +\valeur{a_i} =\vect{x_i}$ pour tout $1\leq i\leq k$),
on dit que $w$ est franchissable à partir de $\vect{x}$, et qu'on a une \emph{exécution} $\rho :\vect{x}\trans{w}{} \vect{y}$.
$\vect{y}$ est alors dit accessible à partir de $\vect{x}$.

De plus, en notant $\valeur{w}\defeq \sum^k_{i=1} \valeur{a_i}$ le vecteur associé à $w$, on obtient $\vect{x} +\valeur{w} = \vect{y}$.
Cependant, l'égalité peut-être vérifiée même si $w$ n'est pas franchissable.
\vspace{3mm}

Nous allons étudier deux ensembles naturellement associés à un VAS $S=(A,\xinit)$ :
\begin{enumerate}
    \item 
    $\lang(S)\defeq \set{w\in A^\ast} {\exists \vect{y}\in\N^d, \xinit\trans{w}{}\vect{y}}$ qui est le \emph{langage} des séquences d'actions franchissables à partir de $\xinit$,
    
    \item $\conf{S}\defeq \set{\vect{y}\in\N^d} {\exists w\in A^\ast, \xinit\trans{w}{} \vect{y}}$ qui est l'ensemble des configurations \emph{accessibles} à partir de $\xinit$. 
\end{enumerate}

% A déplacer ?
\begin{Definition}
Un VAS $S$ est \emph{rationnel} si $\lang(S)$ est rationnel sur $A^\ast$.
\end{Definition}
%----------------------------------------------------------------

\section{Un algorithme de calcul du Graphe de couverture (pour les VAS)}

On étend les configurations des VAS aux vecteurs à coordonnées dans $\Nomega\defeq \N\union\os{\omega}$.
Cela va nous permettre de représenter le graphe des configurations accessibles de manière finie (bien qu'il puisse exister une infinité de configurations accessibles).

Le \emph{graphe de couverture} a pour sommets des configurations de $\Nomega^d$ et pour arêtes des transitions du VAS, étiquetés par une action de $A$.
Il est obtenu en partant d'un sommet initial $s_0:\xinit$ étiqueté par la configuration initiale $\xinit\in \N^d$, puis par récurrence sur la profondeur des noeuds en indiquant les voisins des noeuds accessibles :

Pour chaque noeud $s:\vect{x}$ associé à la configuration $\vect{x}\in \Nomega^d$, on fait partir de $s$ autant d'arêtes que d'actions $a\in A$ qui sont franchissables à partir de $\vect{x}$.
Le sommet d'arrivée de l'arête associée à une action $a$ est déterminé ainsi :
\begin{itemize}
    \item Si $\vect{x}\trans{a}{} \vect{y}$ (déclencher $a$ aboutit à la configuration $\vect{y} := \vect{x} + \valeur{a}$) 
    et qu'il existe un sommet déjà existant $r:\vect{y}$ associé à cette configuration, alors on crée une arête étiquetée par $a$ de $s:\vect{x}$ vers $r:\vect{y}$ ;
    \item Si $\vect{x}\trans{a}{} \vect{y}$ et qu'il existe un ancêtre $r:\vect{z}$ de $s$ (c'est-à-dire tel qu'il existe une chemin dans le graphe déjà créé de $r$ à $s$) avec $\vect{y}>\vect{z}$, 
    alors on crée un nouveau sommet $s':\vect{y'}$ et une arête de $s:\vect{y}$ vers $s':\vect{y'}$ étiquetée par $a$, 
    où $\vect{y'}\in \Nomega^d$ est la configuration de coordonnées $\vect{y'}(i):=\vect{y}(i)$ pour les $1\leq i\leq d$ tels que $\vect{y}(i)=\vect{z}(i)$, et $\vect{y'}(i):=\omega$ si $\vect{y}(i)>\vect{z}(i)$ ;
    \item Si la configuration $\vect{y}$ atteinte n'est pas dans les cas précédents, on crée simplement un nouveau sommet $s':\vect{y}$ et une arête de $s$ à $s'$ étiquetée par $a$.
\end{itemize}

%----------------------------------------------------------------

\section{Une caractérisation pour la rationnalité}

La preuve de décidabilité se divise en deux étapes.
Tout d'abord, on va donner une caractérisation mathématique équivalente à la rationnalité.
On montrera ainsi qu'un VAS est rationnel si et seulement s'il existe une borne sur la décroissance possible des coordonnées des configurations.

\begin{Theorem}\cite{}
    Soit $S=(A,\xinit)$ un VAS.
    Alors $\lang(S)$ est rationnel si et seulement si 
    $$\exists k\in\N, \forall \vect{x}\in\conf{S}, \forall w\in A^\ast, 
    \Big( (\vect{x} +\valeur{w})\in\conf{S} \implies \forall i\in\os{1,\dots,d}, \valeur{w}(i)\geq -k\Big)$$
\end{Theorem}

Une manière équivalente de formuler cette caractérisation serait
$$\exists k\in\N, \forall \vect{x}\in\conf{S}, \forall \vect{y}\in\conf{A,\vect{x}}, \quad
\vect{y}\geq \vect{x} -\vect{k}$$
%	\alain{pourquoi demande-t-on que $\vect{x}\in\conf{S}$ et pas seulement $\vect{x}\in \N^d$ ?}

\alain{pourrait aussi s'écrire avec des fleches: $ \exists k\in\N, \forall \vect{x,y}\in \N^d$ si $\vect{x_0}\trans{*}{} \vect{x} \trans{*}{} \vect{y}$ alors $\vect{y}\geq \vect{x} -\vect{k}$}

où $\vect{k} = (k,k,...,k)\in\N^d$.
\textbf{Méthode de Valk et Vidal-Naquet :}

On prouve d'abord la réciproque en construisant un automate fini reconnaissant $\lang(S)$.
Pour cela, on restreint simplement les états aux configurations de $\os{0,\dots,c}^d$ pour une certaine constante $c\in\N$ obtenue à partir de $k$.
L'autre sens est montré par l'absurde, en s'appuyant sur les circuits dans le graphe de couverture de $S$.


\textbf{Méthode de Ginzburg et Yoeli :}

On introduit une relation $\rel$ sur les configurations de $\conf{S}$ par
$$\vect{x}\rel\vect{y} \ssi \forall w\in A^\ast, 
\Big( \vect{x} +\valeur{w}\in\conf{S} \equivaut \vect{y} +\valeur{w}\in\conf{S} \Big)$$

\alain{Ca a l'air d'être équivalent à $\lang(S,x)=\lang(S,y)$. Est-ce que la relation d'ordre associée $x \sqsubseteq  y$ if $\lang(S,x) \subseteq \lang(S,y)$ est intéressante à regarder ?}

L'objectif est de traduire les caractérisations usuelles de rationnalité dans le modèle des configurations.
La relation $\rel$ correspond en fait à la congruence de Nérode sur le langage $\lang(S)$.
On obtient donc le résultat suivant :

\begin{Theorem}
    Pour un VAS $S$, $\lang(S)$ est rationnel si et seulement si le quotient $\conf{S}/\rel$ est fini.
\end{Theorem}

Preuve : 
Dans un sens, construit l'automate (explicitement) en associant un état à chaque classe d'équivalence.
Dans l'autre, utilise le fait que la congruence de Nérode est connue être finie pour les langages rationnels.

% On peut s'affranchir de la construction de l'automate, et passer directement de la congruence sur les mots à la relation $\rel$.

On prouve ensuite l'équivalence entre $\conf{S}/\rel$ fini et la caractérisation du théorème 1.

%----------------------------------------------------------------

\section{Bornes}

Détaillons maintenant plusieurs propriétés de bornes sur les coordonnées des vecteurs configurations.
On regarde $\os{1,\dots,d}$ l'ensemble des indices des vecteurs codant les configurations d'un VAS $S=(A,\xinit)$.
On peut voir ces coordonnées comme des emplacements accueillant un certain nombre de jetons, qui sont ajoutés ou retirés lors du déclenchement d'une action (lien avec les réseaux de Petri).

Soit $I\subseteq \os{1,\dots,d}$ un sous-ensemble d'indices.
On dira que
\begin{itemize}
    \item $I$ est \emph{borné} pour $\vect{x}\in \Nomega^d$ lorsqu'il existe toujours une coordonnée d'indice dans $I$ qui soit bornée pour toute configuration accessible depuis $\vect{x}$ :
    $$\exists k\in\N, \forall \vect{y}\in\conf{\vect{x}}, \exists i\in I, \vect{y}(i)\leq k$$
    
    \item $I$ est \emph{uniformément borné} pour $\vect{x}\in \Nomega^d$ lorsque toutes les coordonnées d'indice dans $I$ sont bornées pour toute configuration accessible depuis $\vect{x}$ :
    $$\exists k\in\N, \forall \vect{y}\in\conf{\vect{x}}, \forall i\in I, \vect{y}(i)\leq k$$
    
    \item $I$ est \emph{borné inférieurement} pour $\vect{x}\in \Nomega^d$ lorsqu'au moins une coordonnée dans $I$ ne diminue pas plus qu'une certaine borne (même en augmentant les ressources initiales) :
    $$\exists k\in\N, \forall n\in\N, \forall \vect{y}\in\conf{\vect{x} + n\cdot\indicatrice{I}}, \exists i\in I, \vect{y}(i)\geq \vect{x}(i)+n-k$$
    
    \item $I$ est \emph{uniformément borné inférieurement} pour $\vect{u}\in \Nomega^d$ lorsque toutes les coordonnées de $I$ ne décroissent pas plus d'une certaine valeur :
    $$\exists k\in\N, \forall n\in\N, \forall \vect{y}\in\conf{\vect{x} + n\cdot\indicatrice{I}}, \forall i\in I, \vect{y}(i)\geq \vect{x}(i)+n-k$$
\end{itemize}

mes commentaires \\
biblio à commencer\\
vérifier qu'on peut énoncer Vidal-Naquet sur le graphe de couverture minimal défini par le graphe de Karp-Miller dans lequel on a gardé que les marquages maximaux.\\
vérifier que ce nouveau graphe peut être obtenu à partir de Clover en ajoutant les transitions possibles (prolongées par continuité sur $\N^d$). Vérifier qu'il ne manque pas de transitions utiles.
%================================================================

\end{document}